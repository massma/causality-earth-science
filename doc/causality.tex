\documentclass[12pt]{article}
% fonts
\usepackage[scaled=0.92]{helvet}   % set Helvetica as the sans-serif font
\renewcommand{\rmdefault}{ptm}     % set Times as the default text font

% db/ not mandatory, but i recommend you use mtpro for math fonts.
% there is a free version called mtprolite.

% \usepackage[amssymbols,subscriptcorrection,slantedGreek,nofontinfo]{mtpro2}

\usepackage[T1]{fontenc}
\usepackage{amsmath}
\usepackage{amsfonts}
\usepackage{pgf}
% page numbers
\usepackage{fancyhdr}
\fancypagestyle{newstyle}{
  \fancyhf{} % clear all header and footer fields
  \fancyfoot[R]{\vspace{0.1in} \small \thepage}
  \renewcommand{\headrulewidth}{0pt}
  \renewcommand{\footrulewidth}{0pt}}
\pagestyle{newstyle}

% geometry of the page
\usepackage[top=1in,
bottom=1in,
left=1in,
right=1in]{geometry}

% paragraph spacing
\setlength{\parindent}{0pt}
\setlength{\parskip}{2ex plus 0.4ex minus 0.2ex}

% useful packages
\usepackage{natbib}
\bibliographystyle{plainnat}
\usepackage{epsfig}
\usepackage{url}
\usepackage{bm}

\usepackage{titlesec}
\titleformat*{\section}{\normalsize\bfseries}
\titleformat*{\subsection}{\normalsize\bfseries}

\usepackage{enumitem}
\setlist{nolistsep}

\graphicspath{{figs/}}

\begin{document}

\title{Causality for clarification: examples from Earth science}

\author{Adam Massmann\thanks{Corresponding author:
    akm2203@columbia.edu}, Pierre Gentine, Jakob Runge, Elias Bareinboim}

\maketitle
\begin{abstract}
  Use of causal methods is exploding in Earth science. However, most
  applications focus on causal discovery: inferring the causal
  relationships and structure from data. Causal discovery requires
  strong assumptions that are not well justified in generic Earth
  science systems, and can lead to uninterpetable results and
  scientific regress. By focusing on causal discovery we may have
  missed an opportunity to apply causal fundamentals to communicate
  assumptions and organize analyses. We argue that encoding domain
  knowledge in a causal graph, rather than inferring a causal graph
  from data, is a path towards interpretable results and scientific
  progress. We apply causal fundamentals to generic graphs of the
  climate system to identify where causal inference may be most
  tractable in Earth science. Specifically we identify three different
  assumptions under which causal inference may be tractable in generic
  systems: (1) the state space of the system is reconstructable from
  lagged observations of the system; alternatively: (2) the effect of
  interest is only causally affected by the observed portion of the
  state space; alternatively: (3) The cause of interest can be assumed
  to be independent of the evolution of the system’s state. However,
  we also highlight through examples that inclusion of a causal graph
  in any analysis improves communication of assumptions and helps to
  structure analyses, even if causal inference is ultimately
  impossible from the data available.

\end{abstract}

\section{Introduction}


The goal of scientific research is to improve understanding. But, we
cannot improve our understanding with results we cannot interpret, so
we need interpretable results. Interpretation requires clear
communication and justification of assumptions. As we add more tools
to our scientific toolbox, we ought to ask ourselves questions like:

\begin{itemize}
\item ``are the assumptions behind this tool justified?''
\item ``can we interpret the results from this tool?'' (note that
  generally if the assumptions are not justified, we cannot interpret
  the results).
\item ``does this tool help us communicate other assumptions in our analysis?''
\end{itemize}

If the answer to all these questions is ``no'' for our research
application, it might be a sign that using this tool does more to
confuse, rather than clarify, understanding.

This paper is about causality, and is motivated by asking ourselves
the above questions with respect to causal tools used in Earth science
applications. This is particularly relevant because the use of causal
tools is booming across Earth science
\citep[e.g.,][]{ebert-uphoff2012,
  samarasinghe-casuality,runge-causal-timeseries,runge2019inferring,goodwell-causality-2020}
\textbf{could add more}. However, most of the recent use has focused
on causal discovery, or the inference (using data) of causal ``links''
and directions between variables (for example ``does X cause Y?'',
instead of ``how would changing X cause Y to change?''). This focus on
causal discovery is perhaps a missed opportunity: for many
applications in climate science the assumptions behind many causal
discovery algorithms are not well justified, and/or are difficult to
interpret (Section XX, also cite Friedman). We also usually have
strong a priori knowledge of the functional connections in the Earth
system through our knowledge of physics, so have less of a need for
``causal discovery''. However we \textit{are} often interested in
inferring the functional form of causal connections from data (i.e., a
need to infer how changing X would cause Y to change), which is not
addressed with causal discovery. In some sense, we believe that the
current focus on causal discovery in Earth science has us on a path
towards confusion.

The irony of this path to confusion is that some fundamentals of
causal inference are particularly well suited to clarifying
assumptions, \textit{and} in some cases these fundamentals can be used
to calculate the functional form of relationships between Earth
science variables. The main contribution of this paper is to
demonstrate how causal graphs, a fundamental tool of causal inference
introduced in Section XX, can be used to communicate assumptions,
organize analyses, and ultimately improve scientific understanding. We
want to hammer home that pretty much any study could benefit from
inclusion of a causal graph in terms of communication and
clarification, even if in the end the results cannot be interpreted
causally. Causal graphs also encorouge us to think deeply up front
about what we are trying to quantify and how the system is structured,
and can identify infeasible studies early in the research process
(before messing around with data or models), saving us time and
minimizing the chance of (unintentionally) flawed studies. These
points require a fair bit of background and discussion, so we broke
the paper up into sections so that the reader can pick those sections
that are most relevant given their background knowledge and interests:

\begin{itemize}
\item Section XX: Introduce and discuss causal graphs within the
  general philosophy of causality.
\item Section XX: Using a simple relatable example we will explain
  the problem of confounding and how causal graphs can be used to
  isolate the functional mapping between interventions on some
  variable(s) to their effect on other variable(s).
\item Section XX: We draw on past examples from our own research that
  would have benefitted from inclusion of a causal graph, in terms of
  communicating assuptions, and organizing and justifying analyses.
\item Section XX: We turn to more generic examples of graphs that are
  generally constistent with a wide variety of systems in Earth
  science, to highlight some of the difficulties we confront when
  using causal inference in Earth science and how we may be able to
  overcome those difficulties (Section XX).
\item Section XX: A qualitative discussion of causality in Earth
  science, including recommendations for authors on incorporating
  causality into their research, and for reviewers/readers reviewing
  and interpreting causal studies.
\end{itemize}

We hope that this paper will increase understanding of causal graphs
and shift more focus towards using causal fundamentals. We believe
wider adoption will set us on a path towards scientific clarity,
rather than conusion.

\section{What is causality?: the causal graph perspective and its
  usefulness in Earth science}

We view causality through the lens of causal graphs, as introduced in
\citet{pearl1995causal}, mostly because we believe causal graphs are
pragmatically useful in Earth science, rather than because of any
particular philosophical argument for causal graphs as the ``true''
definition of causality.

Causal graphs are directed acyclic graphs that encode our assumptions
about the causal dependencies of a system.  To make a causal graph, a
domain expert simply draws directed edges (i.e. arrows) from variables
that are causes to effects. In other words, to make a causal graph you
just draw a picture (Figure \ref{fig:toy}). Causal graphs are useful
tools because they can be drawn by any domain expert with no required
knowledge of math or probability, but they also represent formal
mathematical objects. Specifically, underlying each causal graph are a
set of equations: each node corresponds to a generating function for
that variable. The function's inputs are all of the node's parents
(plus a random variability term, which we won't worry about for
now). Parents are the other nodes in the graph that point to our node
of interest (e.g., in the simpilest graph X->Y; X is a parent of
Y). So in reality, drawing arrows from ``causes'' to ``effects'' is
really just drawing arrows from function inputs to generating
functions.

In this way, drawing a causal graph is just another way to visualize
and reason about a complicated system of equations, which is a
powerful tool for the Earth scientist: we deal with very complicated
systems of equations and welcome any tool that helps us understand and
reason about their behavior. Some times in Earth sciences we might
know (from physics) the equations for a given function in the causal
graph. However, in practice we often either do not know all of the
functions (e.g., plant stomata response to VPD), or some functions are
computationally intractable to compute (e.g., turbulence, moist
convection, and cloud microphysics in large scale models). In these
scenarios the benefits of causal graphs are fully realized: based on
the causal graph we can calulate from data, using somethign called the
\textit{do-}calculus \citep{pearl-1994-do-calculus}, the functional
relationship between target variables (i.e. effects) and interventions
on any other variables in the graph (i.e. causes). By viewing causal
graphs through this pragmatic lens of calculating the functional form
for functions that we cannot calculate a priori, we simultaneously
identify their value for Earth scientists, while also side stepping
philosophical arguments about the meaning of causality, which are sure
to continue until the heat death of the universe\footnote{and we hope
  they do, we have enjoyed many such arguments in the past}. In Earth
science we often want to know how the system responds to interventions
(e.g., sub grid scale parameterization: how time tendancies respond to
interventions on the large scale state; experimentation: how would
changing land cover from X to Y change the statistics of surface
temperature), and do-calculus provides a way to calculate this reponse
to interventions without relying on approximate numerical models or
(potentially in feasible or unethical) real world experimentation. For
consistency with the causal literature, we will call the response
variables ``effects'', and the intervened upon variables ``causes'',
but we want to reemphisize that for Earth science applications the
exact definition of causality is largely irrelevant (we are
pragmatically interested in cacluling functional forms).

One might say: ``wait, don't we have regression to calculate functions
from data?'' But, as we will see (Section
\ref{sec:causal-graphs-pearls}), uninformed regression is just a
functional mapping of associations between variables, not the response
to interventions. This is the problem that \textit{do-}calculus
solves: it identifies whcih data are needed and how we can use those
data to calculate the functional mapping of interventions, rather than
just associactions that may be attributable to other processes
entirely. Working through an example will hopefully clarify some of
these claims.

\section{A Toy Example: the problem of confounding and the necessity
  of \textit{do-}calculus for calculating interventions}
\label{sec:causal-graphs-pearls}
% intro causal graphs}

We demonstrate the problem of confounding and the necessity of
\textit{do-}calculus we use a very simple toy example involving
clouds, aerosols, and surface solar radiation/sunlight (Figure
\ref{fig:toy}A). Our causal graph consists of:

\begin{figure}
  % % consider \noindent\include... [width=0.75\textwidth]...
  % \includegraphics[height=0.4\textheight]{cloud-aerosol-bidirected.pdf}\\
  \scalebox{1.0}{\input{figs/cloud-aerosol.tex}}
  \caption{A toy graph example to demonstrate basic causal theory,
    involving cloud (C), aerosol (A), and surface solar radiation
    (S). In \textbf{A)} we observe all variables, while in \textbf{B)}
    we do not observe aerosol and we use the semantic of a dashed node
    to denote the unobserved variable. \textbf{C)} presents an
    alternative notation for representing an unobserved or hidden
    common cause that is popular in causal literature \citep[e.g.,
    ``semi-markovian graphs'',][]{shpitser2006}. When using this
    notation it is common practice to only include observed processes
    as nodes in the graph. However in Earth Science we often observe
    much less of the system relative to our knowledge of causal
    dependencies in the system, so we find it more conceptually clear
    to include unobserved common causes explicitly in the graph, and
    prefer the notation in \textbf{B)} to \textbf{C}). Any
    semi-markovian graph can be represented as a graph with unobserved
    nodes, and vice-versa \citep[e.g.,][]{lee2019structural}.}
  \label{fig:toy}
\end{figure}

\begin{enumerate}
\item An edge from aerosols to clouds because aerosols serve as cloud
  condensation nuclei and affect the probability of water vapor
  conversion to cloud.
\item An edge from aerosols to surface solar radiation, because
  aerosols can reflect sunlight back to space and reduce sunlight
  at the surface.
\item An edge from clouds to sunlight, because clouds also reflect
  sunlight back to space and can reduce sunlight at the surface.
\end{enumerate}

Causal graphs encode our assumptions about how the system behaves, and
the nodes and edges that are \textit{missing} from the graph often
represent strong assumptions. For example, in the
cloud-aerosol-sunlight example, clouds also affect aerosols; e.g., by
increasing the likelihood that aerosol will be scavenged from the
atmosphere during precipitation \citep[e.g.,][]{radke-scavenge-1980,
  JURADO20087931, blanco-alegre2018}. By not including an edge from
cloud to aerosol, we are making a strong assumption that we are
ignoring the effect of clouds on aerosols. Considering this example is
intended to be pedagogical for introducing causal theory to the
readers, we will continue with the graph as drawn in Figure
\ref{fig:toy} (Section \ref{sec:necess-cond-caus} explores realistic
Earth system graphs).

Even though mathematical reasoning is not required to construct a
causal graph, the resulting graph encodes specific mathematical
meaning. For example, the graph corresponds to a set fo underlying
functions for each variable:

\begin{align}
  \label{eq:2}
  aerosol &= f(U_{aerosol}) \\
  cloud &= f(aerosol, U_{cloud})\\
  sunlight &= f(aerosol, cloud, U_{sunlight})
\end{align}

where $U$ are random variables due to all the factors not represented
explicitly in the causal graph, and $f$ are deterministic functions
that generate each variable in the graph from their parents and
corresponding $U$.

The presence of the random variables $U$ introduce
a third meaning to the causal graph: they induce a specific
factorization of the joint distrition of the variables into
conditional and marginal facotors:

\begin{equation}
  P(A, C, S) = P(S \, | \,C, A) \, P(C \, | \, A) \, P(A),
\end{equation}

where $A$ represents aerosol, $C$ represents cloud, $S$ represents
surface sunlight/solar radiation, and $P(\cdot | \cdot)$ denotes
conditional probability (Appendix \ref{prob-theory} describes the
notation used in this paper and a brief introduction to probability
theory for unfamiliar readers). The inclusion of randomness in causal
graphs is a powerful tool: by positing a causal graph, we are not
stating that the variables in the graph are the only processes in the
system. Instead, we are stating that all other processes not included
in the graph induce variations in the graph's variables that are
independent of each other (e.g., all $U_{\cdot}$ in Equation
(\ref{eq:2}) are independent). For example, sources of aerosol
variability not considered in Figure \ref{fig:toy}A include
anthropogenic aerosol emission, the biosphere, fires, volcanoes,
etc. \citep[e.g.,][]{Boucher2015}. For cloud, this includes synoptic
forcing, atmospheric humidity,
etc. \citep[e.g.,][]{wallace2006atmospheric}. For radiation, this
includes variability of top of atmosphere radiation,
etc. \citep[e.g.,][]{hartmann2015global}. Figure \ref{fig:toy}A states
that all these external, or \textit{exogenous}, sources of variability
are independent of each other (in very technical terms, this means the
graph is ``\textit{Markovian}'').

We can apply causal graph theory
\citep[e.g.,][]{pearl1995causal,shpitser2006} to the assumptions
encoded in our causal graph to identify which distributions must be
estimated from data in order to calculate the response of effect(s)
(e.g. of sunlight) to an experimental intervention on the cause(s)
(e.g. presence or absence of a cloud). The challenge of causal
inference is to derive the response to the intervention in terms of
only observed distributions. This process of identifying the necessary
observed distributions is formally termed \emph{causal
  identification}. If a causal effect is not identifiable
(\emph{un}-identifiable), for example if calculating a causal effect
requires distributions of variables that we do not observe, then we
cannot use causal inference to calculate a causal effect, even with an
infinite sample of data.

% backdoor path}

A necessary condition for unidentifiability is the presence of an
unblocked backdoor path from cause to effect. Backdoor paths are any
paths going through parents of the cause to the effect. We can block
these paths by selectively observing variables such that no
information passes through them \citep{geiger-d-sep}. If we can
observe variables along the backdoor paths such that they are blocked,
then we have satisfied the \emph{back-door criterion}
\citep{pearl2009} and we can calculate unbiased causal effects from
data.

\begin{figure}
  \input{figs/mutilated-cloud-aerosol.tex}
  \caption{A mutilation of Figure \ref{fig:toy}(A), where we have
    removed directed causal paths from the cause (cloud) to the effect
    (surface solar radiation). There is covariability between cloud
    and surface solar radiation that is not due to the causal
    connection between cloud and surface solar radiation, but is
    instead due to aerosol's role as a common driver.}
  \label{fig:mutilated-toy}
\end{figure}

% backdoor path with example}
Understanding backdoor paths and the backdoor criterion is helped by
example. Returning to our toy example (Figure \ref{fig:toy}A), we
attempt to calculate the causal effect of clouds on sunlight. In other
words, we want to isolate the variability of sunlight due to the
causal link from cloud to sunlight (Figure \ref{fig:toy}A). However,
aerosols affect both cloud and sunlight, so if we naively calculate a
causal effect using correlations between sunlight and cloud, we would
get a biased estimate. To demonstrate this, consider simulated cloud,
aerosol, and sunlight data from a set of underlying equations
consistent with Figure \ref{fig:toy}A and Equation (\ref{eq:2}):


\begin{align}
  aerosol = U_{aerosol} \sim \text{uniform (0, 1]}\\
  cloud = \text{Cloudy if } U_{cloud} + aerosol > 1; \;
  U_{cloud} \sim \text{uniform (0, 1]}\\
  sunlight = &: \begin{cases}
    \text{Cloudy} &: 0.6 \cdot \text{downwelling clear sky radiation}  \\
    \text{Clear} &: \text{downwelling clear sky radiation}
  \end{cases}
\end{align}

where:

\begin{equation*}
  \text{downwelling clear sky radiation} = U_{sunlight} \cdot (1 - aerosol); \;
  U_{sunlight} \sim \text{Normal(340 W m$^{-2}$, 30 \, W m$^{-2}$)}
\end{equation*}

Now, consider not knowing the underlying generative processes, but
instead just passively observing cloud and sunlight. If one were
interested in calculating the effect of cloud on sunlight, and aerosol
data were not available or one were very naive, one approach would be
to bin the data by cloudy and clear conditions and compare the amount
of sunlight between cloudy and clear observations (Figure
\ref{fig:naive-cloud-sunlight}). This approach suggests that clouds
reduce sunlight by, on average, 160 W m$^{-2}$; this is is a strong
overestimation of the true average effect of clouds (-68 W m$^{-2}$),
derived from Equation (\ref{eq:1}). Aerosol induces co-variability
between cloud and sunlight that is unrelated to the causal link from
cloud to sunlight. Graphically, this is clarified by removing all
edges from our cause (cloud) to children of our cause (in this case
sunlight) to create a ``mutilated'' graph (Figure
\ref{fig:mutilated-toy}). We see that clouds are not independent of
surface solar radiation in the mutilated graph; aerosol induces
variability in both cloud and surface solar radiation.  However if
aerosol were fixed (e.g. observed or not varying), cloud and sunlight
would be independent of each other in Figure
\ref{fig:mutilated-toy}. In other words, conditional on aerosol, cloud
and sunlight are independent in the mutilated graph (Figure
\ref{fig:mutilated-toy}), which does not include the causal path from
cloud to sunlight. And, conditional on aerosol in the true system
(Figure \ref{fig:toy}(A)), all co-variability between cloud and
sunlight is only due to the causal edge between cloud and sunlight.
We can mathematically incorporate the requirement that we must
condition on aerosol to isolate the causal effect of cloud on
radiation. Doing so gives the identification of the causal effect of
cloud and aerosol and satisfies the backdoor criterion with
\textit{adjustment} on aerosol:

\begin{figure}
  \includegraphics[]{naiveCloudSunlight.pdf}
  \caption{A naive approach to estimating the ``effect'' of clouds on
    sunlight: bin observations by cloudy and clear day, and compare
    the values of sunlight. This approach yields an average difference
    of 160.43 W m$^{-2}$ between cloudy and clear days, and is a large
    overestimation of the true causal effect of clouds on sunlight
    (-68.0 W m$^{-2}$) in this synthetic dataset.}
  \label{fig:naive-cloud-sunlight}
\end{figure}

\begin{equation}
  P(S | do(C = c)) = \int_{a} P(S \, | \, C = c,
  A=a) \, P(A=a) \; da,
  \label{eq:3}
\end{equation}

where the \textit{do}-calculus \citep{pearl2009} term
($P(S \, | \, do(C\, = \,c))$) represents the probability of sunlight
if we did an experiment where we intervened and set cloud to a value
of our choosing (in this case $c$, which could be ``True'' for the
presence of a cloud, or ``False'' for no cloud). In the case that
observations of aerosols are not available (e.g., Figure
\ref{fig:toy}B, C), our causal effect is not identifiable and we
cannot use causal inference no matter how large the sample size of our
data. This is a powerful research tool: after encoding our domain
knowledge in a causal graph, we can analyze the causal graph and
available observations to determine whether a causal calculation is
possible, \textit{without needing to collect, download, or manipulate
  any data}. For more complicated graphs, causal identification can be
automated \citep{shpitser2006}. We later use this theory to
theoretically assess which general problems are tractable in Earth
science using causal inference (Section \ref{sec:necess-cond-caus}).

Once we have established that a causal effect is identifiable from
data, we must estimate the required observational distributions
(Equation (\ref{eq:3})) from data. Often it may be more computationally
tractable to calculate an average causal effect, rather than the full
causal distribution $P(S | do(C=c))$. Returning to our toy example
(Figure \ref{fig:toy}(A)), the average effect is defined as:

\begin{equation}
  \mathbb{E}(S | do(C = c)) = \int_{s} s \, P(S = s
  | do(C=c)) \, ds,
  \label{eq:4}
\end{equation}

where $\mathbb{E}$ is the expected value. Substituting Equation
(\ref{eq:3}) into Equation (\ref{eq:4}), and rearranging gives:

\begin{equation}
  \mathbb{E}(S | do(C = c))  = \int_{a} P(A=a) \; \mathbb{E}(S \, | \,
  C=c, A=a) \, d a,
  \label{eq:5}
\end{equation}

Where $\mathbb{E}(S \, | \, C=c, A=a)$ is just a regression of sunlight on
cloud and aerosol. Estimating the marginal $P(A)$ is difficult, but
if we assume that our observations are independent and identically
distributed (IID) and we have a large enough sample, we can use the
law of large numbers to approximate Equation (\ref{eq:5}). The law of
large numbers states that if we observe $n$ IID samples of some process, in
this case $A$ (e.g., $a_1$, $a_2$, $\ldots$, $a_n$), then for any
function $f$ \citep{shalizi2013}:

\begin{equation}
  \frac{1}{n} \sum_{i=1}^n f(a_i) \to \int_a P(A=a) f(a) \, d a.
  \label{eq:lln}
\end{equation}

In Equation (\ref{eq:5}), $c$ is fixed by the intervention, so
$\mathbb{E}(S| C=c, A=a)$ is just a function of $a$, and we can use
Equation (\ref{eq:lln}) to justify an approximation to Equation
(\ref{eq:5}) with:

\begin{equation}
  \mathbb{E}(S | do(C = c))  \approx \frac{1}{n} \sum_{i=1}^n
  \mathbb{E}(S \, | \,
  C=c, A=a_i).
  \label{eq:6}
\end{equation}

Data or prior knowledge can inform the regression function for
$\mathbb{E}(S | C=c, A=a_i)$, and as always whatever regression method
is used, it should be checked to insure it is representative of the
data. In the toy example, a linear model conditional on cloud appears
to be a good choice of regression function (Figure
\ref{fig:linear})\footnote{However note that for most real problems in
  Earth science, we probably want to use some type of non linear
  regression (neural networks could be well suited to this
  task).}. The causal effect of clouds on sunlight as calculated using
Equation (\ref{eq:6})) (e.g.
$\mathbb{E}(S | do(C = \text{cloudy})) - \mathbb{E}(S | do(C =
\text{clear}))$) is -68.52 W m$^{-2}$, which closely matches the true
causal effect from Equation (\ref{eq:1}) of -68 W m$^{-2}$. This
example demonstrates how causal inference and theory can be used to
calculate unbiased average effects using regression. Further, causal
inference can be used to justify and communicate assumptions in any
observational analyses employing regression. In the best case, the
causal effect is identifiable from the available observations, and the
regression analysis can be framed as an average causal effect. In the
worst case that identification is not possible from the available
observations, one may present the regression as observed associations
between variables. However, presentation of a causal graph still aids
the reader: the reader can see from the causal graph what the
confounders and unobserved sources of covariability are between the
predictors and the output. In all cases, the presentation of a causal
graph makes explicit the assumptions about the causal dependencies of
the system. Wherever possible, we recommend including causal graphs
with any observation-based analyses.

In summary of the main points of this introduction to causal graphical
models and \textit{do-}calculus:

\begin{itemize}
\item Graphical causal models encode our assumptions about causal
  dependencies in a system (edges are drawn \emph{from} causes
  \emph{to} effects). ``Causal dependencies'' really just refer to
  functional dependencies between inputs (causes) and outputs
  (effect), which are useful in physical sciences.
\item In order to calculate an unbiased causal effect from data, we
  must isolate the covariability between cause and effect that is due
  to the directed causal path from cause to effect. The presence of
  non-causal dependencies between the cause and effect can be deduced
  from the causal graph: the presence of an unblocked backdoor path
  from the cause to the effect leads to non-causal dependencies (and
  co-variation).
\item The backdoor criterion identifies the distributions we must
  calculate from data in order to block all backdoor paths, remove
  non-causal dependence between the cause and effect, and calculate an
  unbiased causal effect from data.
\item The \emph{average} causal effect can be reliably approximated
  with regression (Equation (\ref{eq:6})) derived from the backdoor
  criterion. In this scenario, causal theory and graphs identify the
  variables that should (and should not be) included in the regression
  in order to calculate an unbiased causal effect.
\item Causal identification is a flexible tool that provides the
  distributions that must be estimated from data, while making no
  assumptions about the forms of those distributions. However,
  parametric assumptions can be applied to make the calculation of
  those distributions from data more computationally tractable.
\end{itemize}

\begin{figure}
  \includegraphics[]{aerosolSunlight.pdf}
  \caption{A linear relationship between aerosol and sunlight,
    conditional on cloud. If we use linear regression to calculate the
    average causal effect of cloud on sunlight, as in Equation
    (\ref{eq:6}), our result is very close to the true causal effect
    of -68.0 W m$^{-2}$.}
  \label{fig:linear}
\end{figure}

Here we focused on the backdoor criterion to block backdoor paths. An
un-blockable backdoor path from the cause to the effect is a necessary
condition for unidentifiability. However, it is not sufficient
(e.g. there are other identification strategies like the front door
criterion and instrumental variables that do not rely on observing
variables along the backdoor path). For a complete discussion of
sufficient conditions for unidentifiability we refer you to
\citet{shpitser2006}. For the purpose of identifying generally
tractable causal inference approaches in Earth science (Section
\ref{sec:necess-cond-caus}), we focus the backdoor criterion. Given
that the Earth system evolves continuously according to an underlying
dynamical system and we only partially observe the state space in both
space and time, these other identification strategies are not as
applicable, and an unblockable backdoor path is a sufficient condition
for unidentifiability in the types of graphs that are representative
of Earth science systems (Figure \ref{fig:generic})
\citep{tian2002general}. In other words, blocking backdoor paths is a
requirement for calculating a causal effect in graphs of the form
presented in Figure \ref{fig:generic}.

\section{Causal graphs as communicators, organizers, and time-savers}\label{sec:causal-graphs-as}

In Section \ref{sec:causal-graphs-pearls} we used a toy example to go
through a full causal analysis from drawing the graph, to calculating
the average response of sunlight (the effect) to an intervention on
cloud (the cause). However, often we may not be able to estimate the
causal effects from the available data, as there are serious
challenges with unobserved confounding in generic Earth science
problems (Section \ref{sec:necess-cond-caus}). We want to emphasize,
even if we cannot follow all the way through to calculating
successfully calcualting an interventional effect for our problem of
interest, drawing a causal graph at the beginning of the analysis, and
including it the disemination of results can pay huge dividents in
terms of organization, saving us time, preventing us from
unintentionally executing flawed studiess, and comunicating our
results to the public. Drawing the causal graph at the beginning of
the analysis is not easy: it takes time and deep thought. But, this is
thought that will need to be done at some point in the analysis: we
will need some forumlation in our brain of how the system behaves and
what are the functional dependencies between processes in the
system. Making these dependencies explicit at the beginning of the
analysis forces us to clarify our thinking, exposing factors and
potential problems that we may not have realized until much later (or
never) if we do not do the deep thinking up front. If we realize
issues much later, we will have wasted a lot of time building a flawed
analysis (e.g. downloading data, running models). If we never realize
the issues, then we introduced confusion into the literature. So the
causal graph is both an organizer for scientific analysis, and because
we can draw it at the beginning of the analysis, it is a time saver
preventing us from unwittingly investing in flawed analyses.

Once we have drawn a causal graph, there is no cost to including in in
presentations, papers, and discussions of our results. Making our
assumptions about dependencies in the system explicit greatly improves
the interpretability of our results. Perhaps our analysis and graph
meets the standards for a causal interpretation, but even if it does
not, the causal graph still helps the rest of the community asses the
sources of confounding in teh graph that were not controlled for, and
understand if their conceptualizzation of how the graph is structured
matches the author's assumptions (for rearch without a causal graph,
it is difficult to diagnose if we and the authors even view the system
in the same light).

To support the idea that many analysis would benefit from a causal
graph, we will detail how, if we were to redo a past analysis, we
would start with a causal graph, and the benfits that would have
provided.

\subsection{An example from our past}

In \citet{massmann2017}, the lead author of this paper (Adam)
participated in a field campaign designed to study the impact of
microphysical rain regime (specifically the presence of ice from aloft
falling into orographic clouds) on orographic enhancement of
precipitation. This analysis would have benefited from a causal graph,
and is a nice real example argument for the more common use of causal
graphs are research tools. Our retrospective causal graph of
orographic enhancement in the Nahuelbuta mountains under steady
conditions clearly communicates our assumptions about the system
(Figure \ref{fig:ccope}).

\begin{figure}
  \includegraphics[]{ccope.pdf}
  \caption{A graph representing steady conditions for orographic
    enhancement during the Chilean Coastal Orographic Precipitation
    Experiment \citep[CCOPE,][]{massmann2017}. We were interested in
    the effect of ``rain regime'' on ``orographic enhacement.''
    Observed quantites are represented by solid nodes, while
    unobserved quatnties are represented by dashed nodes. All backdoor
    paths are blocked by observed quantities, so the effect of rain
    regime on orographic enahcement is identifiable.}
  \label{fig:ccope}
\end{figure}

A few things to notice about the graph: many of the variables are
quite general (and even vague) quantities. This can be a useful tool
for communicating assumptions, and we recommend startng with more
general/vague quantities. If logic needs clarifying, the graph can
become more explit (e.g., differentiating wind into speed, direction,
and spatial distribution, both horizontally and vertically). However
the more general graph approaches often yield more intuitive and
interpretable graphs, and are generally easier to manage. ``Wind'',
``stability'', and ``atmospheric moisture'' all refer to upwind
conditions and we assume that these upwind conditions are the relevant
``boundary condition'' for the downwind orographic clouds and
precipitation. This assumptions in ecndoed in teh graph: for example,
if ``wind'' referred to the three dimensional distrubtion of wind over
teh entire domain, then terrain would have to affect wind as well.

An important assumption in this field campaign is that ``synoptic
forcing'' variability operates at larger horizontal spatial scales
(e.g., O(100km)) than the horizontal spatial scales over which the
orography influences physics and kinematics (O(10 km)). This allows us
to assume that upwind ``orographically unaffected'' observations of
synoptic quantities are relatively constant across the mountain
range. So, upwind observations of precipitation at our island and
coastal sites can be thought of as representative of background
``synoptic rain'' across the mountain network. If we define
``orographic enhacement'' as the difference between our mountain sites
and our coastal sites, then orographic enhancement is not directly
affected by synoptic forcing (e.g., we have ``subtracted'' the
synoptic rain out of our orographic metric).

It might be unclear why we are digressing into this discussion of the
choice of orographic enahcement metric: but the importance of this
choice is illuminated by the causal graph. When writing the paper, we
focused on ``orographic difference'' (``mountain sites'' minus
``upwind sites'') as the orographc enahcement metric that should be
used. However, it is common in mountain weatehr research to define
orographic enhacement in terms of the ``orographic ratio'', or the
ratio of mountain sites to upwind sites. Intuitively we felt that the
orographic precipitation metric should be invariant to changes in
background precipitation, which is true for ``orographic difference''
but false for ``orographic ratio.'' However, the causal graph makes
the argument for ``orographic difference'' clear and mathematically
explicit (supporting the ``intuitive'' description). If synoptic
forcing directly impacts our orographic enhacement metric, as is the
case for the ``orographic ratio'', then our causal graph would need an
additional edge from ``synoptic forcing'' to ``orographic
enhancement.'' Because we do not observe ``synoptic forcing'', this
would open up a backdoor path between ``rain regime'' and ``orographic
enhancement,'' so any estimate of the influence of rain regime on
orographic enahcement would be biased. The choice of orographic
enhacement metric matters, because it changes our graph from one where
the effect of interest is identifiable (``orographic difference'') to
one where it is unidenifiable (``orographic ratio''). If we used
orographic ratio, our estimate of warm rain's impact on orographic
enahcement would likely be biased very high: decreased synoptic
forcing decreases the chance of ice falling into orographic clouds
(increasing the likelihood of ``warm rain''), while also decreasing
upwind precipitation (which non-linearly increases the orographic
ratio, up to infinity as it approaches zero). We would falsley
attribute synoptic forcing's effect on orographic ratio to rain
regime. This example shows how causal graphs can provide clear and strong
evidence supporting ideas we ``intuitively know.''

However, perhaps the biggest advantage of including a causal graph in
the CCOPE case would have been in the funding stage. One of the
challenges of the CCOPE field campaign is that the campaign's
designers, Justin Minder, Rene Garreaud and Jefferson Snider, were
trying to do it on a relatively small budget. So, there was no budget
for traditional atmospheric field campaign tools like scanning radars
or aircraft that could measure the three dimensional distribution of
clouds, rain, aerosols, and help quantify important quantites like
``synoptic forcing'' and ``orographic ascent.'' Instead, they proposed
using upwind radiosondes to measure wind, stability, and atmospheric
moisture, and a coastal ground based station to observe aerosol
distribution. By assuming that these measrurements are sufficiently
representative of the ``upwind boundary condition'' for orographic
enhacement over the Nahuelbuta, they were able to observe enough
variables to identify the effect of rain regime on orographic
enhacement (Figure \ref{fig:ccope}). Given the PIs' experience and
thoughtfulness it is no suprise that the field campaign design yeilded
an idenifiable effect. However, inclusion of a causal graph like
Figure \ref{fig:ccope} communicates to proposal reviewers the
assumptions about the systems behavior, and very clearly demonsrtates
that the proposed observation plan allows us to calculate the effect
of interest given the rigorous framework of \textit{do-}calculus. This
type of theoretically-justified framework can really strengthen a
proposal, or alternatively rule out infeasible proposals before one
spends a bunch of time writing and doing backgroudn resaerch. One can
also flip the process and draw the graph first, and then exame the
graph to determine which variables would need to be observed in order
to identify the effect of interest. If we label each node in the graph
with a cost of obtainment (from zero to infinity), we can optimize to
find the set of variables we need to observe to caclulate our effect
of interest such that the total cost of the field experiment is
minimized.

The last important piece of this example is that many other mountain
weather experts may disagree with the causal graph we draw. In our
estimation, this is a good thing - we cannot disagree about things
that we do not know, and we believe that everyone has in their head
some notion of how the system behaves, and making that idea explicit
in a causal graph and allowing ourselves to reconcile that idea with
other's ideas (even if is through disagreement) is a good thing. The
thing we shoudl fear more than disagreement is confusion: and nothing
is more confusing than two researchers who think of a system in a
completely different way, trying to have a discussion about results
based on taht system, without any knowledge that they fundamentally
believe in different structures to the system. Clearing up and
conciling differences in how we view dpendencies of a system seems to
be a productive pre requisite to reconciling contradictory results.

Just one last note - this causal graph would have also benefited the
layout and plan of analysis of the data. The paper itself ended up
being a bit add hoc in how it approached ``controlling'' for wind,
stability, and atmopsheric moisture, mostly just observing if there
are any differences in those quatnties between the two rain
regimes. Part of this was due to a lack of large sample sizes. But,
had we included the causal graph, I thinkwe could have presented a
statistical analysis that was much more clear and motivated by the
graph itself. Also, the small sample size itself is a common problem,
and can be overcome with ``shoe leather'' as David A. Freedman
recommends in his excellent (but at times overly scathing) critizim of
causal infference in the social sciences
(\cite{freedman2010statistical}). However Freedman argues against
causal graphs because on the grounds that we rarely have strong
apriori knowledge of the causal structure of systems of
interset. While this is likely true for the social sciences, in the
physical sciences we usually do have strong apriori knowledge, as
Figure \ref{fig:ccope} demonstrates. Additionally, Freedman misses the
organizational and conceptual benfits of causal graphs, even for
analyses were ``formal'' causal effects are not actually cacluclated:
they can be sueful for communicating assumptions and organzing
analyses. In some ways with CCOPE we did use the shoe leatehr
approach: we identified negligbley small differences in wind,
stability, and atmospheric moisture, and used that logic to suggest
that most of the observed differences in orogrpahic enahcement were
due to rain regime. However, the causal graph would have aided the
communication and focus of that shoe leather work. Also, because not
everythign goes to plan, we ended up hafving trouble with our aerosol
equipment, so that was a possible (unknwon) source of confoudning that
would have been made more clear with icnlsuion of the causal graph.

Which is all a long way to say the causal graphs are useful, they
force us to do a lot of the ``heavy thinking'' up front of our
analysis, which prevent us from wasting a bunch of time and resources
on more mundane research procedures like downloading or collecting
data, and running big, complicated numerical model. With the heavy
thinking done up front and the causal graph already made, we can then
use it to help communicate our results and structure our analysis
(e.g., motivate why we are doing regression, why we are includign the
variables we are in regression, etc.).

\subsection{Summary}

This is one example: but the complexity and structure of causal graphs
will vary by field. Again, we think drawing the graph early will help
clarify and make explicit assumptions (both for ourselves, and also
for the rest fo the communication), and can help identify flawed or
intractable studies. These can also be used as communication tools for
very complicated spatiotemproatl systems, including climate models,
etc. They offer a path to understanding and visualizing inherent
assumptions in any system (for a climate model: what are invariants;
what are the inputs into subgrid parameterizations?). Some of these
graphs might be too hard to visualize on printed papers, but
researchers can link to and use interactive visualization software to
examine compilcated graphs. Another sueful tool for visualizing
complicated graphs are plates \citep{bishop2006pattern}, which can
represent repeated structure like we often get in spatiotemporal
systems. Additionally, as we will see in the next section, we can draw
quite general graphs that are reprentative of many problems in Earth
science.


\section{Assumptions leading to causal identification in Earth
  systems}
\label{sec:necess-cond-caus}

So far we have focused on toy examples, and the utlity of causal
graphs as a communication and organziational tool. However, many may
want to apply a ``full causal analysis'' to an Earth science problem,
so it is useful to examine what common issues we may need encounter,
and how we can overcome them in generic Earth science systems.

Earth science systems are dynamical systems evolving through time
according to an underlying system state
\citep{lorenz-1963,lorenz1996predictability,majda-state}. This offers
both advantages and challenges for causal inference. When constructing
causal graphs we benefit from the temporal ordering of events: we know
that future events can have no causal effect on the past. However,
confounding due to incomplete observation of the system's state space
introduces challenges.

Causal identification and tractable causal inference to Earth science
requires assumptions about the unobserved portions of the state
space. Without such assumptions the unobserved portions of the state
space will introduce confounding for any causal affect of interest
(Figure \ref{fig:generic}a). For example, we generally do not observe
the state space at every time (e.g. $S(t-1/2)$ in Figure
\ref{fig:generic}a), and at any given time, we do not observe the
state space at all locations and for all state variables (e.g. $S(t)$
and $S(t-1)$ in Figure \ref{fig:generic}a). So, if we are interested
in the causal effect of any state variable at time $t$ on some
variable at time $t+1$ (e.g., $E$ in Figure \ref{fig:generic}a), then
the causal effect will be confounded by the unobserved portions of the
state space, and calculating a causal effect is impossible
(un-identifiable) without additional assumptions. We will apply causal
graph theory to identify the additional assumptions that would make
the calculations of causal effects tractable, and may be reasonable in
many Earth science situations.

\begin{figure}
  \input{figs/generic-graph.tex}
  \caption{Generic graphs of the Earth system state sequence, limited
    to a 3 time sequence subset of the infinite sequence. \textbf{A):}
    The reality of the observed earth system: unobserved nodes are
    outlined by dashed lines. We only observe the state space at
    certain times (e.g., no observations at $S(t-1/2)$). At times with
    observations, we only partially observe the full state ($S(t)$,
    $S(t-1)$). In the scenario that we are interested in calculating
    the causal effect of any portion of the state space at time $t$ on
    some effect ($E$) at time $t+1$, the causal effect will be
    confounded by the unobserved portions of the state space, and
    calculating the causal effect is impossible (un-identifiable)
    without additional assumptions. \textbf{B):} The reduction of the
    earth system graph under an assumption that we can reconstruct the
    state $S(t)$ from the observable portions of the state space at
    lagged times $< t$ (see Section
    \ref{sec:stat-reconstr-state}). \textbf{C):} The reduction of the
    earth system graph under an assumption that missing temporal
    observations (e.g. $S(t-1/2)$ in \textbf{A)}) induce independent
    random variations in the cause ($C(t)$) and effect
    ($E(t+1)$). $S'(t)$ denotes the state space, not including the
    cause $C(t)$ (see Sections
    \ref{sec:miss-temp-observ},\ref{sec:observ-port-state}).}
  \label{fig:generic}
\end{figure}

\subsection{Statistical reconstruction of the state space with time
  lagged observations}
\label{sec:stat-reconstr-state}

Takens' theorem implies that we can reconstruct the unobserved
portions of the state space using lagged observations back in time
\citep{takens1981detecting,deyle2011generalized,Sugihara496}. In this
case, the causal graph is greatly simplified (Figure
\ref{fig:generic}b). If we assume that the state is reconstructable,
then we can calculate the causal effect of any reconstructed state on
any future variable or process. The primary advantages of this
approach is that it is the minimum assumption required to calculate a
causal effect in the generic Earth system graph proposed in Figure
\ref{fig:generic}a. However this simplification comes at a cost: we
can only examine the effect of different global states on future
variables, and we cannot examine the causal effect of a specific
variable (e.g. $C(t)$) on another (e.g. $E(t+1)$), because the mapping
from individual observations to a state space reconstruction
transforms the individual observation's importance to additionally
include the unknowable and unobserved portions of the state space. So,
an intervention on any individual observation is of ambiguous meaning
(e.g. we do not know what are the unobserved variables we are
intervening on). Instead, we limit ourselves to only examining the
causal effect of changes in the entire state
holistically. Additionally, this approach requires a shift in
interpretation from the relatively straightforward interventions on
observable variables we have considered thus far to one where we
intervene on the entire state.

For example, consider that we reconstructed our state space from
available observations, and this reconstruction reduced to a discrete
state space of $N$ states, with associated patterns in the observations
at time $\leq t$. If we are interested in the effect of a change in
state on a process at time $t+1$, for example a change from state $1$
to state $2$, this corresponds to an intervention on the
\textit{entire state space at time $t$}. This includes the unobserved
portions of the state space. So, the intervention no longer consists
of just intervening on the observed variables to change them from
observations consistent with state $1$ to state $2$, but instead
intervening on the observed variables \emph{and all unobserved
  variables consistent with the system and changes from state $1$ to
  $2$.}  Conceptualizing unknowable changes in unobserved variables
represents a significant barrier to interpretation, and motivates the
exploration of stronger assumptions that result in a clearer
interpretation (see Section \ref{sec:miss-temp-observ},
\ref{sec:observ-port-state}, \ref{human}).

In practice we also do not know the number of lagged observations that
are required for reconstruction of the state space at time $t$. One
approach is to use the data for guidance. This involves reconstructing
the state at time $t$ with iteratively increasing numbers of lags. We
have confidence we have reproduced the state when the observations at
time $t$ are conditionally independent of each other given the state
reconstruction. However, this approach potentially introduces even
more problems, including but not limited to the assumptions required
in our conditional independence tests. This example highlights the
challenges associated with any method relying on state space
reconstruction with lagged observations: to be sure we reconstruct the
state space we must include observations from a potentially infinite
temporal extent, and there might not be any reliable way to check that
we successfully reconstructed the state space. *relate back to opening
questions: not good on interpretabgle*

\subsection{Missing temporal observations induce independent random
  variations in cause and effect}
\label{sec:miss-temp-observ}

If we can assume that the missing temporal observations
(e.g. $S(t-1/2)$ in Figure \ref{fig:generic}a) induce independent
random variations in the cause and effect, conditional on $S(t-1)$,
then the causal graph reduces to Figure \ref{fig:generic}c. In this
case, if we can reconstruct the state space at time $t-1$ using lagged
observations at time t $\leq t-1$, then we can block all backdoor
paths between our cause $C(t)$ and any future effect $E(t+1)$. In this
case, the causal effect would be calculated as:

\begin{equation}
  P(E(t+1)| do(C(t)=c)) = \int_{S(t-1)} P(E(t+1) \, | \, C(t)=c,
  S(t-1) = s
  )\; P(S(t-1)=s) \, d s,
\end{equation}

where $S(t-1)$ is reconstructed from time lags of observations at
times $\leq t-1$. From the graph it may appear that we could also
block backdoor paths by conditioning on $S'(t)$. However, under these
assumptions $S'(t)$ is incalculable because if we attempt to
reconstruct $S'(t)$ using lagged observations of the state space, the
reconstruction will estimate $S$ rather than $S'$, which includes
$C(t)$ because $C$ is a part of the state space. So, a portion of the
cause's role in the system's evolution will be falsely incorporated
into the reconstruction of the state space, and our causal effect will
be biased.

The assumption that missing temporal observations induce independent
random variations in the cause and effect is required because
otherwise there would be an open backdoor path between the cause and
effect through the unobserved time slice (e.g. at time $t-1/2$ in
Figure \ref{fig:generic}a). Relative to Section
\ref{sec:stat-reconstr-state}, the advantage of this approach to
causal inference is that we can calculate the causal effects of
individual observations (e.g, $C(t)$ on $E(T)$), rather than needing
to interpret the causal impact of the entire state holistically (e.g.,
$S(t)$ on $E(T)$). However, this approach requires an additional
assumption that the missing temporal observations do not induce
dependencies between the cause and effect of interest. As with Section
\ref{sec:stat-reconstr-state}, we must be able to reconstruct the
state space, which may not always be possible, and may also confuse
interpretation. *relate back to opening questions, not good for interpreatble*


\subsection{The unobserved portion of the state space does not affect
  the effect}
\label{sec:observ-port-state}

If we assume that we observe all portions of the state space that
affect the effect, then we can calculate the effect of any state
variable on future processes. In this case, the graph is as in Figure
\ref{fig:generic}c, but $S'(t)$ corresponds to all observations at
time $t$ not including $C$, and so can be used to block all backdoor
paths. If we can assume that there are no interactions between
observations at time $t$ then we can calculate the causal effect as:

\begin{equation}
  P(E(t+1)| do(C(t)=c)) = \int_{S'(t)} P(E(t+1) \, | \, C(t)=c,
  S'(t) = s
  )\; P(S'(t-1)=s) \, d \, s,
\end{equation}

where $S'(t)$ are all observations at time $t$, not including the
cause $C(t)$. However, when blocking backdoor paths with simultaneous
observations it is important to consider whether there are
interactions between observations at time $t$, and whether the
observations are truly simultaneous. Whether there are interactions
between observations is a function of the temporal and spatial extent
of the observations. If observations are instantaneous point
observations indexed in time, then interactions between them can
likely be ignored. However, often the spatial and/or temporal extents
of observations overlap, and in this case an assumption of zero
interactions between observations is not justified. In this case, we
can still block backdoor paths by conditioning on past observations
($S(t-1)$):

\begin{equation}
  P(E(t+1)| do(C(t)=c)) = \int_{S(t-1)} P(E(t+1) \, | \, C(t)=c,
  S(t-1) = s
  )\; P(S(t-1)=s) \, d \, s.
\end{equation}

The assumption that we observe all variables relevant to an effect is
a strong assumption. However, it provides significant benefits both in
terms of the statistical complexity of estimating the causal effect,
as well as in interpretation of the causal effect. We do not need to
reconstruct the state space  using lagged observations, which can be a
statistically and computationally challenging problem. Additionally,
the assumption that we observe everything relevant to the effect is
much easier to interpret relative to an assumption about the degree to
which we can (or cannot) reconstruct the relevant state space with
lagged observations.

\subsection{Applications at the human-Earth system interface: when the
  cause is approximately independent of the system.}
\label{human}

Causal inference becomes substantially more tractable under an
assumption that our causes of interest are independent from the
evolution of the state space (Figure \ref{fig:forcing}). While such a
strong assumption may seem unjustifiable due to tight coupling within
the Earth system, recent historical examples suggest that this
assumption may be applicable to some situations. For example,
coordinated global human response to global warming is relatively
independent of the climate state \citep{arto2014drivers}. We have
failed to reduce green house gas emissions even as global temperature
increased. Instances of reduced rises in global green house gas
emissions are usually due to global economic recession (e.g. the 2008
financial crisis) or pandemic (COVID-19 crisis) rather than factors
directly tied to the climate state. In these examples many global
social, political, health, and economic factors are the primary causes
of global green house gas emission, and while the climate system may
affect these factors, the historical evidence suggests that the
climate system exerts a relatively small impact
\citep{arto2014drivers}.

\begin{figure}
  \includegraphics[]{forcing-graph.pdf}
  \caption{A generic graph asserting an assumption that there are
    forcing external to the evolution of the state-space}
  \label{fig:forcing}
\end{figure}

Generally this logic applies to many systems on the "human-climate"
interface, such as land-use land-cover change in urban centers where
urban planning is relatively independent of recent climate
history. The general graph in Figure \ref{fig:forcing} is applicable
to any analyses for which: 1) we wish to examine the effect of human
behavior on the environment, and 2) we can assume human behavior is
approximately independent of the climate state. Because the climate
state space does not affect the cause (human behavior), there are no
unblocked backdoor paths through the unobserved portions of the state
space. Causal inference is particularly tractable for this class of
problems. *discuss ``human confounding'' (e.g, with covid?*

\section{Reexaming casual discovery in light of what we know}
\label{sec:discovery}

The first question is ``why causal discovery?'' It is not clear; we
usually have strong apriori knowledge of functional dependencies of
our system (e.g., we know the current state is affected by the past
state and will affect teh future, and we usually inow the state
variables at least for physical systems), if not their exact
form. Even if we did not know the dependencies fo the system and
wanted to try to infer them from data, our systems susually do not
meet the assumptions necessary for the algorithm. For example, for the
PC class of algorithms, as well as for ``Granger causality'' we must
assume that we observe all relevant variables of the system (not
usually the case of earth science). We can perhaps assume that for
subsets of the system we obserre verything causally releveant as in
Section \ref{sec:observ-port-state}, but in defining the subset of the
system we have defined the causal graph, so we no longer need to infer it!

Also in practice, for computational tratablility the PC algorithms and
granger causality will invoke linearity assumptiosn, which are not
generally well justified for Earth science systems. Granger causality
assumes that ther eis no underlying state to the system (which is not
the case). Convergent cross-mapping (CCM) \citep{Sugihara496} assumes
that the variables of interst exhibit ``weak couplng'', which is an
untestable assumption that is difficult to interpret (to our
knowledge). The difference in underlying assumptions between Granger
causality and CCM leads to contradictory behavior: when using CCM if X
predicts Y better than Y predictrs X, then Y is the cause of
X. However for Granger causality, if X predicts Y better than Y
predicts X then X is a cause of Y. We should expect that different
assumptions lead to different results, but what is a little alarming
is that they lead to \textit{opposite} results, and it is not clear
(at least to us) when you should use one or ther other, except maybe
we could say we should never use Granger because ther eis always some
underlying state.

In addition to the aforementioned issues with the PC class of
algorithms, we also run into the problem that often multiple possible
graphs could graphs could explain the same observational data. So we
end up with ambiguous links that we need to resolve ...using expert
domian knowledge. Why not use that expert domain knowledge to
construct the graph a priori? If we need to, we can also test the
graph we have drawn with data to see if the data are consistent with
our assumption. THis approach seems much more inline with the
scientific method, and seems much more justified than trying to
massage our domain downledge to match the output of a flawed causal
discovery algorithm.

In the end we are really not sure how to interpret causal discovery
algorithms, and if they can yield useful results in any
field. Statistical experits \citep{freedman2010statistical} are
skeptical of causal discoery's practical ability. However, we view
their general utility as moot, becuase even if they do work in teh
general case, they are only needed when we do not hav apriori
knowledge of the functional dependencies of the system. For Earth
science we often have strong a priori knoweldge fo the functinal
depenedneceies of the system, and we can even draw quite general
graphs that are applicable to a wide range of scenarios (Figure
\ref{fig:generic}).

So then the question is, ``why have causal discoery algorithms seen so
much interest in Eareth science recently?'' One answer could lie with
the perverse incentives we have in academia. Academics are strongly
incentiviezed to publish many articles in peer review. One could argue
that articles with more opaque, and highly technical, assumptions are
more likely to make it in peer reivew (e.g., articles are more likely
to get rejected for assumptions reviews disagree with, than for
articles that rely on assumptions reviewers are unaware of or don't
even really unerstand). The assumptions behind many causal discoery
are genereally very opaque, sometimes even buried in teh
documentation, or worse yet, the code of a given software
pacakge. Also, many of these ``black box'' causal discovery algorithms
are nicely packaged online, so all a resaercher needs to do is
download the existing package and run their data through them, print
soem diagnostic figures, and then craft a story to match the
results. Because these packages are so easily available, the research
themselves need not even really understand the underlyhing
assumptions. So we end up in a situation where there can be a little
cottage industry of relatively easy, mindless analyses using black box
algorithms that are relatively easy to get through peer review. We
also think there is a marketing component; for whatever reason,
``causality'' sounds cool to many people. If one combined a long list
of mindless causal discovery pulbications, with a CV riddled with buzz
terms like ``AI,'' ``ML,'' ``Causality,'' and ``Bayesian,'' and any
researcher can find themselves at the front of the line for the next
open faculty position. If we contrast that to the more fundamental,
domain knowledge driven approach of drawing the causal graph apriori,
we find that a lot of deep (and difficult) thought is required, and we
might find out that many analyses are intractable or flawed, so we
spend more time thinking than publishing, and for those publications
that we do decide to publich, our assumptions are very clearly
communicated in the graph, and so are more likely to be rejected in
peer review. Anyways, this is really a digression into the incentive
structure of academia, which is not where we wanted to be necessarily,
but we do think that we ought to put more focus on quality rather than
quantity when it comes to value in the acadmeic community. This is
also inline with our oroginal argument about interpetability and
communication of assumptions. If we are more rigid abouty requireing
clear communication of assumptions and interpretabiltiy of results, we
might end up with fewer, but higher quality, publciations. A community
wide focus on quality over quantity will also save everyone time: less
time reviewing, less time editing, less time writing, etc.


\section{Conclusions}

In summary, we conclude that:

\begin{itemize}
\item Causal inference from data is a new tool with the potential to
  complement traditional scientific methods, such as numerical and
  real world experimentation. In Earth science, numerical models rely
  on approximations that deviate modeled physics from reality, and
  real world experimentation may be intractable or unethical. Causal
  inference is a third tool to estimate the effect of prescribed
  experimentation on physical processes. Causal inference has a
  different set of advantages and disadvantages, and is a powerful
  complement to numerical and real world experimentation.
\item Causal graphs concisely and clearly encode assumptions about
  causal dependencies between processes. Including a causal graph
  benefits any observational analysis, including those that use
  regression. Depending on the observations available and the causal
  structure of the system, regression analyses can be interpreted as
  an approximation to the average causal effect.
\item Whether or not a causal effect can be calculated from data is
  determined exclusively by the causal graph. Thus the tractability of
  a causal analyses, or the strength of assumptions necessary to make
  an analysis tractable, is determined and assessed before collecting,
  generating, or manipulating data (which can cost a tremendous amount
  in terms of researchers' time, computational resources, and/or
  funding). We recommend early causal analyses to determine
  tractability during a project's conception, before resources are
  spent obtaining or analyzing data.
\item Because the Earth system evolves as a dynamical system through
  time, we can construct broadly applicable, generic Earth system
  causal graphs. However, causal inference in Earth science also
  presents challenges: we only partially observe the state space of
  the system.
\item These challenges can be alleviated by applying causal theory to
  generic causal graphs of the Earth system and identifying the
  assumptions that allow for causal inference from data. These are
  assumptions that:
  \begin{itemize}
  \item The state space of the system is reconstructable
    from lagged observations of the system, as allowed by
    Takens' theorem (Section \ref{sec:stat-reconstr-state}, Section \ref{sec:miss-temp-observ}), or
  \item The effect of interest is only causally
    affected by the observed portion of the state space (Section
    \ref{sec:observ-port-state}), or
  \item The cause of interest can be assumed to be independent of the
    evolution of the system's state (e.g. forcing) (Section
    \ref{human}).
  \end{itemize}
\end{itemize}

Here we focus on the fundamentals of calculating causal effects from
data. However, causal inference is a thriving active area of research,
and there are many other causal inference techniques and abstractions
that would benefit the Earth system research community. For example,
there are techniques for representing variables observed under
selection bias in the causal graph and analyzing whether a causal
effect can be calculated (i.e. identified) given the selection bias
\citep[e.g.,][]{bareinboim2014recovering}. Selection bias is very
relevant in Earth science. For example, satellite observations are
almost always collected under selection bias (e.g. they sample at
certain local times of the day, clouds obscure surface data,
etc.). Additionally, transportability
\citep[e.g.,][]{bareinboim2012transportability} identifies whether one
can calculate a causal effect in a passively observed target domain,
by merging experiments from source domains that may differ from the
target domain. A potential application for transportability in earth
sciences would be to merge numerical model experiments (e.g., global
climate models, cloud models, etc.) and formally transport their
results to the real world. In this case, numerical models are the
source domains that differ from the target domain (``real world'') due
to approximations. Given the limitations of explicit experiments in
Earth science, we hope that causal methods gain wider adoption in
Earth science and that this manuscript provides the necessary
foundation for proper application of causal inference in Earth
science.

\paragraph{Acknowledgments} Thank you to Beth Tellman, James
Doss-Gollin, David Farnham, and Masa Haraguchi for very thoughtful
feedback and comments that greatly improved this manuscript.


\bibliography{references.bib}

\appendix
\section{Basic probability and syntax}
\label{prob-theory}

In this paper we use capital letters to represent random variables
(e.g., ``$X$''). For example, $P(X)$ is the probability distribution
of a random variable $X$. $P(X)$ is a function of one variable that
outputs a probability (or density, in the case of continuous
variables) given a specific value for $X$. We represent specific
values that a random variable can take with lowercase letters (e.g.,
$x$ in the case of $X$). $P(X)$ is shorthand; a more descriptive but
less concise way to write $P(X)$ is $P(X=x)$ which represents the fact
that $P(X)$ is a function of a specific value of $X$, represented by
$x$. We use both notations, and $P(X)$ has the same meaning as
$P(X=x)$.

For the unfamiliar reader, there are a few basic rules and definitions
in probability that provide relatively complete foundations for
building deeper understanding of probability. These are the \textbf{sum rule}:

\begin{equation}
  P(X=x) = \sum_Y P(X=x,\, Y=y)
  \label{eq:sum}
\end{equation}

and the \textbf{product rule}:

\begin{equation}
  P(X=x, \, Y=y) = P(X = x \, | \, Y=y ) P(Y=y) = P(Y = y \, | \, X=x ) P(X=x)
  \label{eq:product}
\end{equation}

The \textit{joint probability distribution} ($P(X=x,Y=y)$) is the
probability that the random variable $X$ equals some value $x$ \emph{and} the
random variable $Y$ equals $y$. The joint distribution is a function
of two variables, $x$ and $y$ which are values in the domains of the
random variables $X$ and $Y$ respectively. The \textit{conditional
  probability distribution} ($p(X = x \, | \, Y=y )$) is also a
function of two variables $x$ and $y$, but it is the probability of
observing $X$ equal to $x$, given that we have observed $Y$ equal to
$y$. In other words, if we filter our domain to only values where
$Y=y$, then $p(X = x \, | \, Y=y )$ is the probability of
observing $X=x$ in this sub-domain where $Y=y$. The \textit{marginal
  probability distribution} ($P(Y=y)$) is just the probability that
$Y$ equals some value $y$, and is a function of only $y$. We can
calculate the marginal probability from the joint distribution by
summing over all possible values values of the other random variables
in the joint (the ``sum rule'' - Equation (\ref{eq:sum})). Additionally,
the joint distribution can factorize into a product of conditional and
marginal distributions (``the product rule'' - Equation
(\ref{eq:product})). These two simple rules can be used to build much of the
theory and applications of probability theory (e.g., Bayes' theorem
$P(Y|X) =\frac{P(X|Y) P(Y)}{P(X)}$). While Equations (\ref{eq:sum})
deals with probability distributions of discrete random variables,
there is also a sum rule analog for continuous random variables and
probability density functions (the syntax of the product rule is the
same):

\begin{equation*}
  P(X=x) = \int_Y P(X=x,\, Y=y) \, dy
\end{equation*}

where $\int_{Y}$ represents an integral over the domain of $Y$ (e.g.,
$\int_{-\infty}^{\infty}$ if $Y$ is a Gaussian random variable).


\end{document}