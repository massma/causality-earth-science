\documentclass[12pt]{article}
\usepackage{float}
% fonts
\usepackage[scaled=0.92]{helvet}   % set Helvetica as the sans-serif font
\renewcommand{\rmdefault}{ptm}     % set Times as the default text font

% db/ not mandatory, but i recommend you use mtpro for math fonts.
% there is a free version called mtprolite.

% \usepackage[amssymbols,subscriptcorrection,slantedGreek,nofontinfo]{mtpro2}

\usepackage[T1]{fontenc}
\usepackage{amsmath}
\usepackage{amsfonts}
\usepackage{pgf}
% page numbers
\usepackage{fancyhdr}
\fancypagestyle{newstyle}{
  \fancyhf{} % clear all header and footer fields
  \fancyfoot[R]{\vspace{0.1in} \small \thepage}
  \renewcommand{\headrulewidth}{0pt}
  \renewcommand{\footrulewidth}{0pt}}
\pagestyle{newstyle}

% geometry of the page
\usepackage[top=1in,
bottom=1in,
left=1in,
right=1in]{geometry}

% paragraph spacing
\setlength{\parindent}{0pt}
\setlength{\parskip}{2ex plus 0.4ex minus 0.2ex}

% useful packages
\usepackage{natbib}
\bibliographystyle{plainnat}
\usepackage{epsfig}
\usepackage{url}
\usepackage{bm}

\usepackage{titlesec}
\titleformat*{\section}{\normalsize\bfseries}
\titleformat*{\subsection}{\normalsize\bfseries}

\usepackage{enumitem}
\setlist{nolistsep}

\graphicspath{{figs/}}

\begin{document}

% \title{Causality for clarification: examples from Earth science}
\title{Causal inference for process understanding in Earth Sciences}

\author{Adam Massmann\thanks{Corresponding author:
    akm2203@columbia.edu}, Pierre Gentine, Jakob Runge, Elias Bareinboim}

\maketitle
\begin{abstract}
  There is growing interest in the study of causal methods in the
  Earth sciences. However, most applications have focused on causal
  discovery, i.e. inferring the causal relationships and structure
  from data. This paper looks at causality through the lens of causal
  {\it inference} and examines how expert-defined causal graphs, a
  fundamental from causal theory, can be used to clarify assumptions,
  identify tractable problems, and aid interpretation of results and
  their causality in Earth science research. We apply causal theory to
  generic graphs of the Earth system to identify where causal
  inference may be most tractable and useful to address problems in
  the Earth science and avoid potentially incorrect
  conclusions. Specifically, causal inference may be particularly
  powerful when: (1) the effect of interest is only causally affected
  by the observed portion of the state space; or: (2) The cause of
  interest can be assumed to be independent of the evolution of the
  system’s state; or: (3) the state space of the system is
  reconstructable from lagged observations of the system. However, we
  also highlight through examples that inclusion of a causal graph in
  any analysis provides explicit definition and communication of
  assumptions and hypotheses and helps to structure analyses, even if
  causal inference is ultimately challenging given the data
  availability, limitations and uncertainties.
\end{abstract}

\paragraph{Note:} This is a living document. We will update this
manuscript as our understanding of causality's role in Earth science
research evolves. Comments, feedback, and edits are enthusiastically
encouraged, and we will add acknowledgements and/or coauthors as we
receive community contributions! (to edit the manuscript directly you
can fork and submit a pull request at
\url{https://github.com/massma/causality-earth-science}, or you can
alternatively email us comments and suggestions and we will
incorporate them).

\section{Introduction}

There is growing interest in the study of causal inference methods in
the Earth sciences \citep[e.g.,][]{salvucci2002, ebert-uphoff2012,
  kretschmer2016,
  samarasinghe2020,runge-causal-timeseries,runge2019inferring,goodwell-causality-2020}. However,
most of this work focuses on causal discovery, or the inference (using
data) of causal structure: ``links'' and directions between
variables. In some cases, causal discovery can be used to estimate the
structure of a causal graph and the relationships between variables
when the graph is not known a priori. However in many if not most
Earth system applications, the causal graph is already known based on
physical insights. For instance the impact of El Ni\~{n}o on West
American rainfall is known to be causal and the graph does not need to
be discovered.

This paper looks at causality through a different, but complementary,
lens and examines how assumed causal graphs \citep{pearl1995causal}, a
fundamental from causal theory, can be used to clarify assumptions,
identify tractable problems, and aid interpretation of results in
Earth science research. Our goal is to distill
\citep[e.g.,][]{olah2017} the basics of the graphical approach to
causality in a way that is relatable for Earth scientists,
\textbf{hopefully motivating more widespread use and adoption of
  causal graphs to organize analyses and communicate
  assumptions}. These tools are relevant now more than ever, as the
abundance of new data and novel analysis methods have inevitably led
to more opaque results and barriers to the communication of
assumptions.

Beyond their usefulness as communication tools, if certain conditions
are met, causal graphs can be used to calculate, from data, the
generalized functional form of relationships between Earth science
variables \citep{pearl2009causality}. Ultimately, deriving generalized
functional relationships is a primary goal of science. While we know
the functional relationships between some variables a priori, there
are many relationships we do not know \citep[e.g., ecosystem scale
water and carbon fluxes;][]{massmann2019, zhou2019arid,
  zhou2019feedback, grossiord2020}, or that we do know but are
computationally intractable to calculate \citep[e.g., clouds and
microphysics at the global scale:][]{randall2003, gentine2018,
  zadra2018, gagne2020emulation}. In these types of applications,
causal graphs give us a path toward new scientific knowledge and
hypothesis testing: generalized functional relationships that were
inaccessible with traditional tools.

The main contribution of this paper is to demonstrate how causal
graphs, a fundamental tool of causal inference introduced in Section
\ref{sec:what-caus-caus}, can be used to communicate assumptions,
organize analyses and hypotheses, and ultimately improve scientific understanding and replicability. We
want to emphasize that almost any study could benefit from inclusion
of a causal graph in terms of communication and clarification of hypotheses, even if
in the end the results cannot be interpreted causally. Causal graphs
also encourage us to think deeply in the initial stages of analysis
about hypotheses and how the system is structured, and can
identify infeasible studies early in the research process before is being time
spent on analysis, acquiring data, building/running models,
etc. These points require some background and discussion, so the
paper is divided into the following sections:

\begin{itemize}
\item Section \ref{sec:what-caus-caus}: Introduce and discuss causal
  graphs within the general philosophy of causality and applications
  in Earth science.
\item Section \ref{sec:causal-graphs-pearls}: Using a simple relatable
  example we explain the problem of confounding and how causal graphs
  can be used to isolate the functional mapping between interventions
  on some variable(s) to their effect on other variable(s).
\item Section \ref{sec:causal-graphs-as}: We draw on a past example
  that benefits from inclusion of a causal graph, in terms of
  communicating assumptions, and organizing and justifying analyses.
\item Section \ref{sec:necess-cond-caus}: We turn to more generic
  examples of graphs that are generally consistent with a wide variety
  of systems in Earth science, to highlight some of the difficulties
  we confront when using causal inference in Earth science and how we
  may be able to overcome these challenges.
\end{itemize}

Throughout the discussion key terms will be emphasized in italics.

\section{The graphical
  perspective to causality and its usefulness in the Earth
  sciences}\label{sec:what-caus-caus}

While there are many different definitions and interpretations of
``causality'', for this manuscript we view causality through the lens
of causal graphs, as introduced in \citet{pearl1995causal} and
discussed more extensively in \citet{pearl2009causality}. We take this
perspective because we believe causal graphs are useful in Earth
science, rather than because of any particular philosophical argument
for causal graphs as the ``true'' representation of causality.

\begin{figure} \includegraphics[]{cloud-aerosol.pdf}
  \caption{A toy graph example to demonstrate basic causal theory,
    involving cloud (C), aerosol (A), and surface solar radiation
    (S).}
  \label{fig:toy}
\end{figure}

Causal graphs are Directed Acyclic Graphs (\emph{DAG}s) that encode
our assumptions about the causal dependencies of a system. To make a
causal graph, a domain expert simply draws directed edges
(i.e. arrows) from variables that are causes to effects. In other
words, to make a causal graph you draw a diagram summarizing the
assumed causal links between variables (e.g., Figure
\ref{fig:toy}). Causal graphs are useful tools because they can be
drawn by domain experts with no required knowledge of maths or
probability, but they also represent formal mathematical
objects. Specifically, underlying each causal graph are a set of
equations called \emph{structural causal models}: each node
corresponds to a generating function for that variable, and the inputs
to that function are the node's parents. Parents are the other nodes
in the graph that point to a node of interest (e.g., in the most
simple graph $X \to Y$; $X$ is a parent of $Y$). So in reality,
drawing arrows from ``causes'' to ``effects'' is synonymous with
drawing arrows from function inputs to generating functions.

In this way, drawing a causal graph is another way to visualize and
reason about a complicated system of equations, which is a very useful
tool for the Earth scientist: we deal with complicated interacting
systems of equations and welcome tools that help us understand and
reason about their collective behavior. In some cases we may know a
priori (from physics) the equations for a given function in a causal
graph. However, in practice we often either do not know all of the
functions a priori \citep[e.g., plant stomata response to
VPD;][]{massmann2019, zhou2019arid, zhou2019feedback, grossiord2020},
or some functions are computationally intractable to compute
\citep[e.g., turbulence, moist convection, and cloud microphysics in
large scale models;][]{zadra2018,gentine2018}. In these scenarios the
benefits of causal graphs are fully realized: based on the causal
graph we can calculate from data, using the \textit{do-}calculus
\citep{pearl-1994-do-calculus}, the response of target variables
(i.e. effects) to \textit{interventions} on any other variables in the
graph (i.e. causes). When combined with statistical modelling (i.e.,
regression), one can estimate the functional relationship between
interventions on causes and effects.

By viewing causal graphs through this pragmatic lens of calculating
the functional form of relationships that we do not know a priori, we
simultaneously identify casual graphs' value for Earth scientists
while also side stepping philosophical arguments about the meaning of
causality. Causal graphs are pragmatic because in the Earth sciences
we often need to estimate how the system responds to
\emph{interventions} (prescribed changes to variables of interest, or
``causes''). For example, for sub-grid physical parameterization in Earth system models we need to
estimate the time tendencies' response to \emph{interventions} on the
large scale state. We also may desire to calculate experiments: for
example how changing land cover from forest to grasslands affects (the
statistics of) surface temperature. \textit{Do-}calculus provides a
way to calculate this response to interventions without relying on
approximate numerical models or real world experimentation, which can
be infeasible or unethical \citep[as is the case for geoengineering;
e.g., unilateral decisions to seed the oceans with iron, or spray
aerosols in the atmosphere,][]{hamilton2013no}. While we want to
maintain this emphasis on causality as a method for calculating the
generalized response to intervention (possibly using regression to
calculate the functional form of that response), for consistency with
the causal literature we will call the response variables ``effects'',
and the intervened upon variables ``causes''.

For some, it may not be clear how the functional response to
\emph{interventions} is different from naive regression between
observed variables. We will demonstrate in Section
\ref{sec:causal-graphs-pearls} how uninformed regression is just a
functional mapping of associations between variables, and how this
differs from the response to interventions. This is the problem that
\textit{do-}calculus solves: it identifies which data are needed and
how we can use those data to calculate the functional mapping of
interventions, rather than just associations that may be attributable
to other processes entirely. Because \emph{interventions} generalize
much better than associations, \textit{do-}calculus is especially
relevant for scientists and engineers. Working through an example will
clarify some of these claims.

\section{A Toy Example: the problem of confounding and the necessity
  of \textit{do-}calculus for calculating interventions}
\label{sec:causal-graphs-pearls}
% intro causal graphs}


To demonstrate the problem of \textit{confounding} and the necessity
of causal graphs/\textit{do-}calculus, we use a very simple toy
example involving clouds, aerosols, and surface solar
radiation/sunlight. As shown in Figure \ref{fig:toy}, the causal
graph consists of:

\begin{enumerate}
\item An edge from aerosols to clouds because aerosols serve as cloud
  condensation nuclei and affect the probability of water vapor
  conversion to cloud.
\item An edge from aerosols to surface solar radiation, because
  aerosols can reflect sunlight back to space and reduce sunlight at
  the surface.
\item An edge from clouds to sunlight, because clouds also reflect
  sunlight back to space and can reduce sunlight at the surface.
\end{enumerate}

Causal graphs encode our assumptions about how the system behaves, and
the nodes and edges that are \textit{missing} from the graph often
represent strong assumptions about the lack of functional
dependence. For example, in the cloud-aerosol-sunlight example, clouds
also affect aerosols; e.g., by increasing the likelihood that aerosol
will be scavenged from the atmosphere during precipitation
\citep[e.g.,][]{radke-scavenge-1980, jurado2008,
  blanco-alegre2018}. By not including an edge from cloud to aerosol,
we are making the assumption that we are neglecting the effect of
clouds on aerosols, which is not generally true. Considering this
example is intended to be pedagogical for introducing causal theory to
the readers, we will continue with the graph as drawn in Figure
\ref{fig:toy} (Section \ref{sec:necess-cond-caus} explores realistic
Earth system graphs, and we refer the reader to \cite{gryspeerdt-2019}
for a more realistic causal treatment of aerosols and clouds).

% Elias recommended removing this: place discussion of cycles
% somehwere else?, and also preventing the graph from containing any
% cycles (a path from a variable to itself) which is a requirement of
% the theory: graphs must by acyclic. This acylic requirement may
% raise concerns for the reader; many problems in Earth science
% contain feedbacks that introduce cycles. However, any feedback can
% be represented as an acyclic graph by explicitly resolving the time
% evolution of the feedback in the graph (Section
% \ref{sec:necess-cond-caus} contains examples of such graphs)

Even though mathematical reasoning is not required to construct a
causal graph, the resulting graph encodes specific causal meaning
based on qualitative, physical understanding of the system. For
example, the graph corresponds to a set of underlying functions,
called a structural causal model, for each variable:

\begin{align}
  \label{eq:2}
  aerosol &\leftarrow f_{aerosol} (U_{aerosol}) \\
  cloud &\leftarrow f_{cloud} (aerosol, U_{cloud})\\
  sunlight &\leftarrow f_{sunlight} (aerosol, cloud, U_{sunlight})
\end{align}

where $U$ are random variables due to all the factors not represented
explicitly in the causal graph, and $f$ are deterministic functions
that generate each variable in the graph from their parents and
corresponding $U$.

The presence of the random variables $U$ introduces a third meaning to
the causal graph: they induce a specific factorization of the joint
distribution between variables into conditional and marginal factors:
\begin{equation}
  P(A, C, S) = P(S \, | \,C, A) \, P(C \, | \, A) \, P(A),
\end{equation}
where $A$ represents aerosol, $C$ represents cloud, $S$ represents
surface solar radiation, and $P($X$ | $Y$)$ denotes conditional
probability of $X$ given $Y$ (Appendix \ref{prob-theory} describes the notation used in
this paper and a brief introduction to probability theory for
unfamiliar readers). The inclusion of randomness in causal graphs is a
key tool: by positing a causal graph, we are not stating that the
variables in the graph are the only processes in the system nor that the relationships are deterministic. Instead,
we are stating that all other processes not included in the graph
induce variations in the graph's variables that are \emph{independent} of
each other (e.g., all $U_{\cdot}$ in Equation (\ref{eq:2}) are
independent). For example, sources of aerosol variability not
considered in Figure \ref{fig:toy} include anthropogenic aerosol
emission, the biosphere, fires, volcanoes,
etc. \citep[e.g.,][]{Boucher2015}. For cloud, this includes synoptic
forcing or atmospheric humidity,
etc. \citep[e.g.,][]{wallace2006atmospheric}. For radiation, this
includes variability of top of atmosphere radiation,
etc. \citep[e.g.,][]{hartmann2015global}. Figure \ref{fig:toy} states
that all these external, or \textit{exogenous}, sources of variability
are independent of each other (in very technical terms, this means the
graph is ``\textit{Markovian}'').

We can now apply causal inference theory
\citep[e.g.,][]{pearl1995causal,shpitser2006} to evaluate whether the
assumptions encoded in our causal graph to identify which
distributions must be estimated from data in order to calculate the
response of effect(s) (e.g. of sunlight) to an experimental
intervention on the cause(s) (e.g. presence or absence of a cloud)
(Alternative: We can now apply causal inference theory
\citep[e.g.,][]{pearl1995causal,tian2002general} to evaluate whether
the assumptions encoded in the causal graph can identify which
distributions must be estimated from data in order to calculate the
response of effect(s) (e.g. of sunlight) to an experimental
intervention on the cause(s) (e.g. presence or absence of a
cloud).). The goal of causal inference is to derive the response to
the intervention in terms of only observed distributions. This process
of identifying the necessary observed distributions is formally termed
\emph{causal identification} \citep[][, Ch. 3]{pearl2009causality}. If
a causal effect is not identifiable (\emph{un}-identifiable), for
example if calculating a causal effect requires distributions of
variables that we do not observe, then we cannot use causal inference
to calculate a causal effect, even with an infinite sample of
data. However even then effects may be estimable if we are willing to
make further assumptions, for example about the functional form and
noise terms or structure of the hidden common causes (CITE - JAKOB?).

A necessary condition for \emph{unidentifiability} is the presence of
an unblocked \emph{backdoor path} from cause to effect \citep[][,
Ch. 3]{pearl2009causality}. Backdoor paths are any paths going
through parents of the cause to the effect. We can block these paths
by selectively observing variables such that no information passes
through them \citep{geiger-d-sep}. If observations are not available
for the variables required to block the path, the path will be
\emph{unblocked}. However, if we can observe variables along the
backdoor paths such that they are all blocked, then we have satisfied
the \emph{back-door criterion} \citep{pearl2009} and we can calculate
unbiased causal effects from data.

Understanding backdoor paths and the backdoor criterion is helped by
an example. Returning to our toy example (Figure \ref{fig:toy}), we
attempt to calculate the causal effect of clouds on sunlight. In other
words, we want to isolate the variability of sunlight due to the
causal link from cloud to sunlight (Figure \ref{fig:toy}). However,
aerosols affect both cloud and sunlight (i.e., there is a backdoor
path from cloud to aerosol to sunlight), so if we naively calculate a
causal effect using correlations between sunlight and cloud, we obtain
a biased estimate. To demonstrate this, consider simulated cloud,
aerosol, and sunlight data from a set of underlying equations
consistent with Figure \ref{fig:toy} and Equation (\ref{eq:2}):


\begin{align}
  \begin{split}
    \text{aerosol} =& \; U_{aerosol}; \; U_{aerosol} \sim
    \text{uniform (0, 1]}\\ \text{cloud} =& \; \text{Cloudy if } U_{cloud} +
    \text{aerosol} > 1; \; U_{cloud} \sim \text{uniform (0, 1]}\\ \text{sunlight}
    =& \begin{cases} \text{Cloudy} &: 0.6 \cdot \text{downwelling clear
        sky radiation} \\ \text{Clear} &: \text{downwelling clear sky
        radiation}
    \end{cases}
    \label{eq:1}
  \end{split}
\end{align}

where:

\begin{equation*} \text{downwelling clear sky radiation} =
  U_{sunlight} \cdot (1 - aerosol); \; U_{sunlight} \sim
  \text{Normal(340 W m$^{-2}$, 30 \, W m$^{-2}$)}
\end{equation*}

Now, consider not knowing the underlying generative processes, but
instead just passively observing cloud and sunlight. If one were
interested in calculating the effect of cloud on sunlight, and aerosol
data were not available, one direct but incorrect approach would be to bin the data
by cloudy and clear conditions and compare the amount of sunlight
between cloudy and clear observations (Figure
\ref{fig:naive-cloud-sunlight}). This approach would suggest that
clouds reduce sunlight by, on average, 158 W m$^{-2}$; this is is a
strong overestimation of the true average effect of clouds (-68 W
m$^{-2}$), derived from Equation (\ref{eq:1}). This overestimation is
due to aerosol-induced co-variability between cloud and sunlight that
is unrelated to the causal link from cloud to sunlight. However if
aerosols were constant (e.g. observed or not varying), any
co-variability between cloud and sunlight would be attributable to the
causal edge between cloud and sunlight (Figure \ref{fig:toy}(A)). In
other words, conditional on aerosol, all co-variability between cloud
and sunlight is only due to the causal effect of cloud on sunlight.
We can mathematically encode this requirement that we must condition
on aerosol to isolate the causal effect of cloud on radiation, and
doing so identifies the causal effect of cloud on sunlight by
satisfying the backdoor criterion with \textit{adjustment} on aerosol:

\begin{figure} \includegraphics[]{naiveCloudSunlight.pdf}
  \caption{A naive approach to estimating the ``effect'' of clouds on
    sunlight: bin observations by cloudy and clear day, and compare the
    values of sunlight. This approach yields an average difference of
    157.74 W m$^{-2}$ between cloudy and clear days, and is a large
    overestimation of the true causal effect of clouds on sunlight (-68.0
    W m$^{-2}$) in these synthetic data.}
  \label{fig:naive-cloud-sunlight}
\end{figure}

\begin{equation} P(S | do(C = c)) = \int_{a} P(S \, | \, C = c, A=a)
  \, P(A=a) \; da,
  \label{eq:3}
\end{equation} where the \textit{do}-expression ($P(S \, | \, do(C\, = \,c))$) represents the probability of sunlight
if we did an experiment where we intervened and set cloud to a value
of our choosing (in this case $c$, which could be ``True'' for the
presence of a cloud, or ``False'' for no cloud). In the case that observations of aerosols are not available, our causal effect is not identifiable and
we cannot generally use causal inference no matter how large the
sample size is of our data.

The \emph{DAG} is therefore a powerful analysis tool: after encoding
our domain knowledge in a causal graph, we can analyze the available
observations to determine whether a causal calculation is possible,
\textit{without needing to collect, download, or manipulate any
  data}. For more complicated graphs, causal identification can be
automated \citep[][ and \url{http://www.dagitty.net/},
\url{https://causalfusion.net}]{tian2002general,shpitser2006,huang2006identifiability,Bareinboim7345,
  textor2017,}. We later use this theory to theoretically assess which
general problems are tractable in Earth science using causal inference
(Section \ref{sec:necess-cond-caus}).

Once we have established that a causal effect is identifiable from
data, we must estimate the required observational distributions
(Equation (\ref{eq:3})) from data. Often it may be more
computationally tractable to calculate an average causal effect,
rather than the full causal distribution $P(S | do(C=c))$. Returning
to our toy example (Figure \ref{fig:toy}(A)), the average effect is
defined as:

\begin{equation} \mathbb{E}(S | do(C = c)) = \int_{s} s \, P(S = s |
  do(C=c)) \, ds,
  \label{eq:4}
\end{equation}

where $\mathbb{E}$ is the expected value. Substituting Equation
(\ref{eq:3}) into Equation (\ref{eq:4}), and rearranging gives:

\begin{equation} \mathbb{E}(S | do(C = c)) = \int_{a} P(A=a) \;
  \mathbb{E}(S \, | \, C=c, A=a) \, d a,
  \label{eq:5}
\end{equation}

Where $\mathbb{E}(S \, | \, C=c, A=a)$ is just a regression of
sunlight on cloud and aerosol. Estimating the marginal $P(A)$ is
difficult, but if we assume that our observations are independent and
identically distributed (IID) and we have a large enough sample, we
can use the law of large numbers to approximate Equation
(\ref{eq:5}). The law of large numbers states that if we observe $n$
IID samples of some process, in this case $A$ (e.g., $a_1$, $a_2$,
$\ldots$, $a_n$), then for any function $f$ \citep{shalizi2013}:

\begin{equation} \frac{1}{n} \sum_{i=1}^n f(a_i) \to \int_a P(A=a)
  f(a) \, d a.
  \label{eq:lln}
\end{equation}

In Equation (\ref{eq:5}), $c$ is fixed by the intervention, so
$\mathbb{E}(S| C=c, A=a)$ is just a function of $a$, and we can use
Equation (\ref{eq:lln}) to justify an approximation to Equation
(\ref{eq:5}) with:

\begin{equation} \mathbb{E}(S | do(C = c)) \approx \frac{1}{n}
  \sum_{i=1}^n \mathbb{E}(S \, | \, C=c, A=a_i).
  \label{eq:6}
\end{equation}

Data or prior knowledge can inform the regression function for
$\mathbb{E}(S | C=c, A=a_i)$, and as always whatever regression method
is used, it should be checked to ensure it is representative of the
data. In this simple example, a linear model conditional on cloud
should be a suitable choice for the regression function (Figure
\ref{fig:linear})\footnote{However, most real problems in Earth
  science require non-linear approximation methods like neural
  networks and/or advanced machine learning methods; for examples, we
  refer readers to \citep{bishop2006pattern}.}. The causal effect of
clouds on sunlight as calculated using Equation (\ref{eq:6})) (e.g.
$\mathbb{E}(S | do(C = \text{cloudy})) - \mathbb{E}(S | do(C =
\text{clear}))$) is -67.69 W m$^{-2}$, which closely matches the true
causal effect from Equation (\ref{eq:1}) of -68 W m$^{-2}$. This
example demonstrates how causal inference and theory can be used to
calculate unbiased average effects using regression, subject to the
assumptions clearly encoded in the causal graph. Further, causal
inference can be used to justify and communicate assumptions in any
observational analyses employing regression. In the best case, the
causal effect is identifiable from the available observations, and the
regression analysis can be framed as an average causal effect. In the worst case that identification is not possible
from the available observations, one may present the regression as
observed associations between variables. (Alternative: In the
case that identification is not obtainable, one could first present
the regression as observed associations between variables and should
make it clear that causation cannot be established given the available
assumptions.) However, presentation of a
causal graph still aids the reader: the reader can see from the causal
graph what the confounders and unobserved sources of covariability are
between the predictors and the output. In all cases, the presentation
of a causal graph makes explicit the assumptions about the causal
dependencies of the system. Wherever possible, we recommend including
causal graphs with any observation-based analyses.

In summary of the main points of this introduction to causal graphical
models and \textit{do-}calculus:

\begin{itemize}
\item Graphical causal models encode our assumptions about causal
  dependencies in a system (edges are drawn \emph{from} causes
  \emph{to} effects). ``Causal dependencies'' really just refer to
  functional dependencies between inputs (causes) and outputs
  (effect), which are useful in the Earth sciences. (Alternative:
  Graphical causal models encode the assumptions the scientist is
  willing to make about the functional dependencies in the system
  under investigation (edges are drawn from causes to
  effects). Naturally, these dependencies are between the inputs
  (causes) and outputs (effect) in the system, which are both not only
  natural but also useful in the physical sciences.)
\item In order to calculate an unbiased causal effect from data, we
  must isolate the covariability between cause and effect that is due
  to the directed causal path from cause to effect. The presence of
  non-causal dependencies between the cause and effect can be deduced
  from the causal graph: the presence of an \textit{unblocked backdoor
    path} from the cause to the effect leads to non-causal
  dependencies (and co-variation).
\item The \emph{backdoor criterion} identifies the variables that we
  must condition on in order to block all backdoor paths, remove
  non-causal dependence between the cause and effect, and calculate an
  unbiased causal effect from data.
\item The \emph{average} causal effect can be reliably approximated
  with a regression (Equation (\ref{eq:6})) derived from the backdoor
  criterion. In this scenario, causal theory and graphs identify the
  variables that should be included in the regression in order to
  calculate an unbiased causal effect (however researchers should
  still ensure their choice of regression model is appropriate for the
  data). While not demonstrated by our example, causal theory and
  graphs identify the variables that \textit{should not} be included
  in the regression: we can also bias a causal effect by
  including too many variables in a regression.
\item The do-calculus and identification theory provide a flexible
  tool to determine whether an effect is identifiable and, if so,
  which distributions should be estimated from data, while making no
  assumptions about the the forms of the underlying functions and
  distributions. However, parametric assumptions can be applied to
  make the calculation of those distributions from data more
  computationally tractable.
\end{itemize}

\begin{figure} \includegraphics[]{aerosolSunlight.pdf}
  \caption{A linear relationship between aerosol and sunlight,
    conditional on cloud. If we use linear regression to calculate the
    average causal effect of cloud on sunlight, as in Equation
    (\ref{eq:6}), our result is very close to the true causal effect of
    -68.0 W m$^{-2}$.}
  \label{fig:linear}
\end{figure}

Here we focused on the \emph{backdoor criterion} to block backdoor
paths. An unblocked backdoor path from the cause to the effect is a
necessary condition for unidentifiability. However, an unblocked
backdoor path from the cause to the effect is not a sufficient
condition for unidentifiability: there are other identification
strategies like the front door criterion and instrumental variables
that do not rely on observing variables along the backdoor path
\citep{pearl2009causality}, and can be used in some cases where
observations are not available to satisfy the backdoor criterion. We
focus on the backdoor criterion because it is the most fundamental and
direct method for adjusting for confounding, the most intuitive for an
introduction to causality, and is the most relevant for the generic
temporal systems present in the Earth sciences (Figure
\ref{fig:generic}, also see \citet{tian2002general} for more
discussion on sufficient conditions for unidentifiability). However,
causal identification through other methods like instrumental
variables and the front door criterion can also be automated; we refer
the reader to \citet{pearl2009causality} for further discussion and
software tools like \url{http://www.dagitty.net/} and
\url{http://www.causalfusion.net} for interactive exploration.

(Alternative: Even though we have focused on the backdoor criterion,
we note that the existence of an unblockable backdoor path is not
sufficient for unidentifiability. In fact, there exist other
identification strategies like the front door criterion and
instrumental variables that do not rely on observing variables along
the backdoor path [Pearl, 2009b], and can be used in both
non-parametric and parametric settings. Still, the backdoor criterion
is one of the most fundamental and direct method for adjusting for
confounding, the most intuitive for an introduction to causality, and
is one of the most relevant for the generic temporal systems present
in the Earth sciences (Figure 6), which we’ll discuss later on; see
also Tian and Pearl [2002] for a more detailed discussion on
sufficient conditions for unidentifiability.

As mentioned earlier, other causal identification methods like the
front door criterion and instrumental variables can also be automated;
we refer the reader to [Pearl, 2000; Bareinboim and Pearl, 2016] for
further discussion, and software tools like
http://www.causalfusion.net for interactive exploration.)


\section{Beyond toys: causal graphs as communicators, organizers, and
  time-savers}\label{sec:causal-graphs-as}

In Section \ref{sec:causal-graphs-pearls} we used a toy example to
demonstrate a causal analysis starting with drawing a graph and ending
with the successful calculation of the average response of sunlight
(the effect) to an intervention on cloud (the cause). However, often
we may not be able to ultimately estimate the causal effects from the
available data. In many cases there are serious challenges due to
unobserved confounding in generic Earth science problems (Section
\ref{sec:necess-cond-caus}), and for other cases we may lack enough
samples to estimate the necessary distributions (or regressions) with
sufficient certainty (e.g., Equation (\ref{eq:6})). However, we want
to emphasize that, even if calculating a causal effect might be in
some cases impossible, drawing a causal graph at the beginning of an
analysis offers tremendous benefits in terms of organization and
communication of the hypotheses being tested. Investing time to reason
about the functional structure of our problem at the outset can save
us time in the long term, forcing us to clarify our thinking early,
expose potential challenges, and identify intractable approaches.

Additionally, once the causal graph is drawn, we can use it as a
communication tool and include it in presentations, papers, and
discussions of our results. Making our assumptions about dependencies
in the system explicit greatly improves the interpretability and
reproducibility of our results. Perhaps our analysis and graph meet
the standards for a causal interpretation, but even if they do not,
the causal graph helps the rest of the community asses the sources of
confounding in the graph that were not controlled for, and understand
if their conceptualization of the graph structure matches the authors'
assumptions. Often there are more assumptions being made than are
communicated, and even when they are communicated, the assumptions do
not always get discussed in a precise way. Including a causal graph
allows the assumptions to be clearly known, and discussed in a precise
and rigorous way \citep{hannart-da}.

To support the idea that many analyses would benefit from a causal
graph, we will detail how a past project benefits from a causal
graph. This example also moves beyond the toy example of Section
\ref{sec:causal-graphs-pearls}, and demonstrates causal graphs'
applicability to real problems in Earth science.

\subsection{A real Example: Measuring The Impact of Microphysical Rain
  Regime on Orographic Enchancement of Precipitation (Alternative: An
  example from our past)}

In \citet{massmann2017}, the lead author of this manuscript
participated in a field campaign designed to study the impact of
microphysical rain regime (specifically the presence of ice from aloft
falling into orographic clouds) on orographic enhancement of
precipitation. This field campaign and analysis benefits from a causal
graph and is a nice real-world example argument for the more common
use of causal graphs as research tools. Our retrospective causal graph
of orographic enhancement in the Nahuelbuta mountains under steady
conditions clearly communicates our assumptions about the system
(Figure \ref{fig:ccope}).

\begin{figure} \includegraphics[]{ccope.pdf}
  \caption{A graph representing steady conditions for orographic
    enhancement during the Chilean Coastal Orographic Precipitation
    Experiment \citep[CCOPE,][]{massmann2017}. We were interested in the
    effect of ``rain regime'' on ``orographic enhancement.''  Observed
    quantities are represented by solid nodes, while unobserved quantities
    are represented by dashed nodes. All backdoor paths are blocked by
    observed quantities, so the effect of rain regime on orographic
    enhancement is identifiable.}
  \label{fig:ccope}
\end{figure}

Many of the variables in the graph, such as ``synoptic forcing'',
``wind'', and ``orographic flow'', are quite general
quantities. Keeping quantities general can lead to more intuitive and
interpretable graphs by limiting the number of details and nodes that
one must store in their mind. However, if logic needs clarifying, the
graph can become more explicit (e.g., differentiating wind into speed,
direction, and spatial distribution, both horizontally and
vertically). As graphs become more complicated, one can leverage
interactive visualization software, or use static graph abstractions
like plates \citep{bishop2006pattern}, which are particularly well
suited to representing repeated structure common to spatiotemporal
systems.

While the exact details of the graph and the assumptions it encodes
(e.g., ``wind'', ``stability'', and ``atmospheric moisture'' all refer
to upwind conditions, and we assume that these upwind conditions are
the relevant ``boundary condition'' for the downwind orographic clouds
and precipitation) are interesting, a particularly noteworthy feature
of the graph is that the field campaign's effect of interest, rain
regime on orographic enhancement, is identifiable from the field
campaign's observations. This is subject to the assumptions encoded in
the graph, but those assumptions are explicitly represented and
communicated by the graph. The causal graph helps interpret the field
campaign's results, and in some sense proves (or could disprove) that
the design of the field campaign is sound.

For field campaigns, causal graphs are particularly useful at the
planning and proposal stage. Such a causal graph could be included in
any field campaign proposal, improving communication about the system
and also rigorously justifying the campaign’s observations as
necessary for calculating the desired effect(s). Even before the
proposal, one could start with a causal graph, and then analyze it to
determine which observations are needed to meet the campaign’s
goals. Building on this idea, one could attach costs associated with
observing each variable in the graph, and automatically determine the
set of observations that minimizes cost while still allowing us to
calculate our effect(s) of interest.

While this is just one example, it demonstrates that causal graphs are
useful beyond toy problems in the Earth system. Additionally, as we
will see in Section \ref{sec:necess-cond-caus}, we can draw quite
general graphs that are representative of many problems in Earth
science. We hope the reader considers drawing a causal graph as a
first step in their next project; they help structure, organize, and
clarify our analysis and its assumptions.


\section{Overcoming unobserved confounding and partial observation of Earth system state}
\label{sec:necess-cond-caus}

(Remove?: So far we have focused on toy (Section
\ref{sec:causal-graphs-pearls}) and very specific (Section
\ref{sec:causal-graphs-as}) examples). We now turn our attention to
more general and generic problems in Earth science systems, the common
challenges we may encounter when attempting causal inference, and how
we can overcome these challenges. (Alternative: We now turn our
attention to more general and generic problems in Earth science
systems. In particular, we will investigate common challenges that one
may encounter in practice when attempting to perform causal inference
when some variables are unobserved.)

Earth science systems are (typically) dynamical systems evolving
through time according to an underlying system state
\citep{lorenz-1963,lorenz1996predictability,majda-state}. This offers
both advantages and challenges for causal inference. When constructing
causal graphs we may benefit from the temporal ordering of events
\citep{runge2019inferring}: we know that future events can have no
causal effect on the past. However, confounding due to incomplete
observation of the system's state variables introduces challenges;
challenges that are not unique to this paper's causal lens: incomplete
observation of the system precludes many ``causal discovery''
algorithms as well (see \citet{runge2019inferring} for a detailed
review).

Causal identification and tractable causal inference in Earth science
requires assumptions about the unobserved portions of the state space
and how these unobserved portions of the state space affect observed
variables. (Alternative: Causal identification and tractable causal
inference in Earth science requires assumptions about the underlying
system and how unobserved portions of the state affect the observed
ones.) Without such assumptions the unobserved portions of the state
space will introduce confounding for any causal affect of interest
(Figure \ref{fig:generic}a). For example, we generally do not observe
the state space at every time (e.g. $S(t-1/2)$, Figure
\ref{fig:generic}a), and at any given time, we do not observe the
state space at all locations and for all state variables (e.g. $S(t)$
and $S(t-1)$ in Figure \ref{fig:generic}a). In other words, despite
our impressive and growing array of satellite, remote sensing, and in
situ observation systems, we are still very far from observing every
relevant state variable at every location in time and space.  So, if
we are interested in the causal effect of any state variable at time
$t$ on some variable at time $t+1$ (e.g., $E$ in Figure
\ref{fig:generic}a), then the causal effect will be confounded by the
unobserved portions of the state space, and calculating a causal
effect is impossible (un-identifiable) without additional
assumptions.

\begin{figure} \input{figs/generic-graph.tex}
  \caption{A generic graph of the Earth system state
    sequence. Unobserved nodes are outlined by dashed lines. We only
    observe the state space at certain times (e.g., no observations at
    $S(t-1/2)$), and at times with observations, we only partially
    observe the full state ($S(t)$, $S(t-1)$). In the scenario that we
    are interested in calculating the causal effect of any portion of
    the state space at time $t$ on some effect ($E$) at time $t+1$,
    the causal effect will be confounded by the unobserved portions of
    the state space, and calculating the causal effect is impossible
    (un-identifiable) without additional assumptions.}
  \label{fig:generic}
\end{figure}

What are some assumptions or scenarios that allow us to overcome this
problem of unobserved confounding due to partial observation of the
state space? There are several assumptions we can make that may be
reasonable for many generic scenarios in Earth science, which we
elaborate upon below. For each assumption, we draw a graph, elaborate
on the identification formula and how average effects can be estimated
(for example, using regression to esimtate $\mathbb{E}(\cdot )$), and
include some strategies for testing the assumption with data.

% reformulate this as table, with columns: assumption, graph,
% identification formula, commentary, check.
\paragraph{Assumption: we observe all state variables that impact our effect(s) of interest}

\begin{figure}[H]
  \includegraphics[]{observe-everything.pdf}
  \caption{A generic graph for the assumption that we observe all
    state variables that impact our effect(s) of interest. $C(t)$ is
    the cause of interest, $E(t+1)$ is the future effect of interest,
    and $O(t)$ are all observations not including $C(t)$. The full
    state $S(t-1)$ is not observed (dashed node).}
  \label{fig:observe-everything}
\end{figure}

\begin{itemize}
\item \textbf{Graph:} Figure \ref{fig:observe-everything}.
\item \textbf{Commentary:} While we may not observe the entire state
  of the system, we somtimes may assume that we observe the portion of
  the state space that affects our specific variable of interst. In
  this case, we can calculate causal effects by blocking backdoor
  paths with the observed portion of the state space.
\item \textbf{Identification formula:}
  \begin{equation*}
    P(E(t+1) \, | \, do(C(t) = c)) = \int_{o} P(E(t+1) \, | \, C(t) = c,
    O(t) = o) \, P(O(t)=o) \; d o,
  \end{equation*}
\item \textbf{Average causal effect:}
  \begin{equation*}
    \mathbb{E}(E(t+1) \, | \, do(C(t) = c)) \approx \frac{1}{n}
    \sum_{i=1}^n \mathbb{E}(E(t+1) \, | \, C(t)=c, O(t)=o_i),
  \end{equation*}
  where $C(t)$ is the cause of interest, $E(t+1)$ is the effect of
  interest, and $O(t)$ are all observed variables.
\item \textbf{Check:} There is no way to check this assumption with
  data. Therefore, this assumption requires strong physical
  justification well supported by the literature. Care is also
  required to insure that there are no interactions between $C(t)$ and
  $O(t)$; e.g., the observations at a given time are truly
  ``simultaneous'' and cannot causally affect each other.
\end{itemize}

\newpage

\paragraph{Assumption: We can reconstruct the state at any given time
  using lagged observations of the system}

\begin{figure} \includegraphics[]{reconstruction.pdf}
  \caption{A generic graph for the assumption that we can reconstruct
    the state at any given time using lagged observations of the
    system (S' is the reconstructed state).}
  \label{fig:reconstruction}
\end{figure}

\begin{itemize}
\item \textbf{Graph:} Figure \ref{fig:reconstruction}.
\item \textbf{Commentary:} While we may not observe the entire state
  space, in some cases we may be able to reconstruct the state at any
  given time using lagged observations of the system \citep[see
  Takens' theorem,][]{takens1981detecting}. In this case, we can
  examine the impact of changes in reconstructed state on any future
  variable of interest, or also use the reconstructed state to block
  backdoor paths and examine the effect of individual variables on
  future variables.
\item \textbf{Identification formula:}
  \begin{equation*}
    P(E(t+1) \, | \, do(C(t) = c)) = \int_{s} P(E(t+1) \, | \, C(t) = c,
    S'(t-1) = s) \, P(S'(t-1)=s) \; d s.
  \end{equation*}
\item \textbf{Average causal effect:}
  \begin{equation*}
    \mathbb{E}(E(t+1) \, | \, do(C(t) = c)) \approx \frac{1}{n}
    \sum_{i=1}^n \mathbb{E}(E(t+1) \, | \, C(t)=c, S'(t-1)=s_i),
  \end{equation*}
  where $C(t)$ is the cause of interest, $E(t+1)$ is the effect of
  interest, and $S'(t-1)$ is the reconstructed state using lagged
  observations.
\item \textbf{Check:} A good check to determine if we were successful
  in reconstructign the state space is to test whether the observed
  variables are conditionally independent, given the reconstructed
  state variable. This approach bears similarity to the deconfounder
  approach introduced by \cite{yixin-2019}, which argues that causal
  effects can be calculated for any problem where we can infer a
  latent variable that renders the causes conditionally
  independent.
\end{itemize}

\newpage

\paragraph{Assumption: the cause of interest may be independent of the
  systems' state evolution}

\begin{figure} \includegraphics[]{forcing-graph.pdf}
  \caption{A generic graph asserting an assumption that there are
    forcings external to the evolution of the state-space.}
  \label{fig:forcing}
\end{figure}

\begin{itemize}
\item \textbf{Graph:} Figure \ref{fig:forcing}
\item \textbf{Commentary:} In some cases, we may assume that
  \textbf{the cause of interest is independent of the systems' state
    evolution}. While not generally true, in some cases a variable may
  behave independent of the system's state while still causally
  affecting that state. For example, some human behavior may be
  approximated as independent of the climate state (e.g., city
  planning and land use decisions).
\item \textbf{Identification formula:}
  \begin{equation*}
    P(E(t+1) \, | \, do(C = c)) = P(E(t+1) \, | \, C = c)
  \end{equation*}
\item \textbf{Average causal effect:}
  \begin{equation*}
    \mathbb{E}(E(t+1) \, | \, do(C = c)) = \mathbb{E}(E(t+1) \, | \, C=c)
  \end{equation*}
\item \textbf{Check:} To check this assumption, we can test if the
  cause/forcing is independent of the past state. If the cause is
  independent of the past state, we have stronger confidence that the
  assumption holds.
\end{itemize}

While this list is certainly not exhaustive,
it presents some promising approaches that may apply to many
scenarios. Also while the provided checks present some strategies that
might indentify when assumptions break down, there is no general way
to ``validate'' assumptions using data. Phsyical and science-based
justification and reasoning are ultimately always required.

\section{Conclusions}

In summary, causal graphs and causal inference is a powerful way to reason about problems in the Earth system. Specifically, this review aimed at showing that:

\begin{itemize}
\item Causal graphs concisely and clearly encode assumptions about
  causal/functional dependencies between processes. Including a causal
  graph benefits any observational or modeling analysis, including
  those that use regression to better specify the underlying hypotheses.
\item Whether a causal effect can be calculated from data is
  determined by the causal graph. Thus the tractability of a causal
  analysis, or the strength of assumptions necessary to make the
  analysis tractable, is determined and assessed before collecting,
  generating, or manipulating data (which can cost a tremendous amount
  in terms of researchers' time or computational resources). We
  recommend early causal analyses to determine tractability during a
  project's conception, before resources are spent obtaining or
  analyzing data.
\item Calculated causal effects measure the response of target
  variables (i.e. effects) to \textit{interventions} on other
  variables in the system (i.e. causes). With statistical modeling
  (i.e., regression) one can estimate the functional relationship
  between interventions on causes and effects.  These functional
  relationships \textit{generalize} because they map
  \textit{interventions} onto the response: unlike observed
  associations (e.g., standard regression), we know the response is
  attributable to the function's input, and not some other process in
  the system (e.g., a hidden common cause). This causal approach opens
  up a new path to calculate generalized functional relationships when
  we either do not know the functional form a priori, or it is too
  computationally intractable to calculate from models.
\item Because the Earth system and its constituents are dynamical systems through time, we can construct broadly applicable, generic Earth system
  causal graphs. We can use these to calculate generalized functional
  relationships between processes of interest, which would not be
  possible with associations/correlation/naive regression. However,
  causal inference in Earth science also presents challenges: we only
  partially observe the state space of the system.
\item These challenges can be alleviated by applying causal theory to
  generic causal graphs of the Earth system and identifying the
  assumptions that allow for causal inference from data (Section
  \ref{sec:necess-cond-caus}).
\end{itemize}

Here we focus on the fundamentals of calculating causal effects from
data. However, causal inference is a thriving active area of research,
and there are many other causal inference techniques and abstractions
that could benefit the Earth system research community. For example,
there are techniques for representing variables observed under
selection bias in the causal graph and analyzing whether a causal
effect can be calculated (i.e. identified) given selection bias
\citep[e.g.,][]{bareinboim2014recovering,correa2018generalized}. Selection
bias is very relevant in Earth sciences. For example, satellite
observations are almost always collected under selection bias
(e.g. clouds obscure surface data and they sample at certain local
times of the day which is connected to top of atmosphere solar
forcing). Additionally, transportability
\citep[e.g.,][]{bareinboim2012transportability,Bareinboim7345,lee2019general}
identifies whether one can calculate a causal effect in a passively
observed target domain, by merging experiments from source domains
that may differ from the target domain. A potential application for
transportability in earth sciences would be to merge numerical model
experiments (e.g., Earth system models) and formally transport their
results to the real world. In this case, numerical models are the
source domains that differ from the target domain (``real world'') due
to approximations and different resolutions.

However, because of the speed and scope of this recent growth in
causal theory, applied domains lag in establishing these recent
theoretical developments' utility for applied analysis. We believe
that the use of causal graphs to organize and structure analyses is
mature and directly applicable to many applications, and can serve as
a gateway to these more sophisticated (and less mature) causal
methods. We hope that drawing and including causal graphs in Earth
science research becomes much more common, and that this manuscript
provides some of the necessary foundation for readers to feel
comfortable using causal graphs in their research.

\paragraph{Acknowledgments} The authors want to than Beth Tellman,
James Doss-Gollin, David Farnham, and Masa Haraguchi for thoughtful
feedback and comments that greatly improved an earlier version of this
manuscript.


\bibliography{references.bib}

\appendix
\section{Basic probability and syntax}
\label{prob-theory}

In this paper we use capital letters to represent random variables
(e.g., ``$X$''). For example, $P(X)$ is the marginal probability distribution
of a random variable $X$. $P(X)$ is a function of one variable that
outputs a probability (or density, in the case of continuous
variables) given a specific value for $X$. We represent specific
values that a random variable can take with lowercase letters (e.g.,
$x$ in the case of $X$). $P(X)$ is shorthand; a more descriptive but
less concise way to write $P(X)$ is $P(X=x)$ which represents the fact
that $P(X)$ is a function of a specific value of $X$, represented by
$x$. We use both notations, and $P(X)$ has the same meaning as
$P(X=x)$.

For the unfamiliar reader, there are a few basic rules and definitions
in probability that provide relatively complete foundations for
building deeper understanding of probability. These are the
\textbf{sum rule}:

\begin{equation} P(X=x) = \sum_Y P(X=x,\, Y=y)
  \label{eq:sum}
\end{equation}

and the \textbf{product rule}:

\begin{equation} P(X=x, \, Y=y) = P(X = x \, | \, Y=y ) P(Y=y) = P(Y =
  y \, | \, X=x ) P(X=x)
  \label{eq:product}
\end{equation}

The \textit{joint probability distribution} ($P(X=x,Y=y)$) is the
probability that the random variable $X$ equals some value $x$
\emph{and} the random variable $Y$ equals $y$. The joint distribution
is a function of two variables, $x$ and $y$ which are values in the
domains of the random variables $X$ and $Y$ respectively. The
\textit{conditional probability distribution} ($p(X = x \, | \, Y=y
)$) is also a function of two variables $x$ and $y$, but it is the
probability of observing $X$ equal to $x$, given that we have observed
$Y$ equal to $y$. In other words, if we filter our domain to only
values where $Y=y$, then $p(X = x \, | \, Y=y )$ is the probability of
observing $X=x$ in this sub-domain where $Y=y$. The \textit{marginal
  probability distribution} ($P(Y=y)$) is just the probability that $Y$
equals some value $y$, and is a function of only $y$. We can calculate
the marginal probability from the joint distribution by summing over
all possible values values of the other random variables in the joint
(the ``sum rule'' - Equation (\ref{eq:sum})). Additionally, the joint
distribution can factorize into a product of conditional and marginal
distributions (``the product rule'' - Equation
(\ref{eq:product})). These two simple rules can be used to build much
of the theory and applications of probability theory (e.g., Bayes'
theorem $P(Y|X) =\frac{P(X|Y) P(Y)}{P(X)}$). While Equations
(\ref{eq:sum}) deals with probability distributions of discrete random
variables, there is also a sum rule analog for continuous random
variables and probability density functions (the syntax of the product
rule is the same):

\begin{equation*} P(X=x) = \int_Y P(X=x,\, Y=y) \, dy
\end{equation*}

where $\int_{Y}$ represents an integral over the domain of $Y$ (e.g.,
$\int_{-\infty}^{\infty}$ if $Y$ is a Gaussian random variable).


\end{document}