\documentclass[12pt]{article}
% fonts
\usepackage[scaled=0.92]{helvet}   % set Helvetica as the sans-serif font
\renewcommand{\rmdefault}{ptm}     % set Times as the default text font

% db/ not mandatory, but i recommend you use mtpro for math fonts.
% there is a free version called mtprolite.

% \usepackage[amssymbols,subscriptcorrection,slantedGreek,nofontinfo]{mtpro2}

\usepackage[T1]{fontenc}
\usepackage{amsmath}
\usepackage{amsfonts}
\usepackage{pgf}
% page numbers
\usepackage{fancyhdr}
\fancypagestyle{newstyle}{
  \fancyhf{} % clear all header and footer fields
  \fancyfoot[R]{\vspace{0.1in} \small \thepage}
  \renewcommand{\headrulewidth}{0pt}
  \renewcommand{\footrulewidth}{0pt}}
\pagestyle{newstyle}

% geometry of the page
\usepackage[top=1in,
bottom=1in,
left=1in,
right=1in]{geometry}

% paragraph spacing
\setlength{\parindent}{0pt}
\setlength{\parskip}{2ex plus 0.4ex minus 0.2ex}

% useful packages
\usepackage{natbib}
\bibliographystyle{plainnat}
\usepackage{epsfig}
\usepackage{url}
\usepackage{bm}

\usepackage{titlesec}
\titleformat*{\section}{\normalsize\bfseries}
\titleformat*{\subsection}{\normalsize\bfseries}

\usepackage{enumitem}
\setlist{nolistsep}


\begin{document}

\title{Causal inference in earth science}

\author{Adam Massmann\thanks{Corresponding author: akm2203@columbia.edu}}

\maketitle

\begin{abstract}
  Causal inference from data is a third tool for scientific discovery,
  complementing numerical and real world experimentation. We examine
  the necessary assumptions and theoretical justification for applying
  causal inference methods to generic earth system scenarios. The
  primary challenge of using causal inference in earth science is that
  we generally only partially observe the state space of the earth
  science system. However an advantage of earth science systems is
  that we generally know the temporal ordering of events and the speed
  with which events' influence propagates through the system. These
  characteristics can be used to filter events which can or cannot
  influence other events in an earth science system, and identify
  sufficient conditions for calculating causal effects from data. We
  apply causal theory to generic earth science systems to show that
  causal inference is identifiable from data when either: 1) we can
  reconstruct the unobserved state space of the system using
  time-lagged observations; or 2) we can assume causes are independent
  of the larger earth system (occasionally justiable at the
  human-climate interface). If neither of these two conditions are
  met, then causal effects are confounded by the unobserved portions
  of the state space, and causal inference is not possible, even with
  an infinite supply of data. /techincally I show necessary conditons
  for un-identfiability, not sufficient contions. see shpitser and
  pearl 2006, can either meet those conditions or just put a note and
  loosen up my statements/.
\end{abstract}

\section{Introduction}

% usefullness causal inference e.s.

Causal inference is the estimation of effects
(e.g. evapotranspiration) in response to changes in causes (e.g. soil
moisture) using passive observations of a system. Causal inference is
an alternative path towards scientific discovery, which traditionally
relies on experimentation (e.g. numerical, lab, or real world) to
determine the effect of varying causes. In earth sciences in
particular it may be impossible or unethical to execute real world
experiments, and numerical experiments may rely on approximations that
bias causal estimates relative to the real world. Given these
challenges in earth science, causal inference from data is a
tantilizing tool that allows for robust estimatation of causal effects
that may be otherwise impossible using traditional tools.

% challenges of ci in e.s.}

However, causal inference from data presents its own challenges; the
causal effect must be identifiable from the set of observable
variables. Whether or not a causal effect is identifiable is
determined from our assumptions about how a system behaves and which
variables we can obtain observations for. If an effect is determined
to be unidentifable, then it is impossible to calculate the causal
effect of interest, even with an infitinte sample of data. Specific to
problems in earth science, the identification of causal effects is
challenging given two fundamental properties of many generic earth
science problems: the system evolves through time according to an
underlying dynamical system, and we only ever partially observe the
state space of that dynamical system. Using graph theory developed in
\citep{pearl1995causal} we determine the generic problems causal
inference is justified for, and the necesarry assumptions for applyign
causal inference. There is the potential for applying causal inference
in two scenarios: (1) under an assumption that the cause is
independent of the system's state space (e.g. at the human-climate
interface, anthropogenic green house gas emission, land use change)
(Section \ref{human}); or (2) under certain conditions where we can
estimate the state space of the system from using carefully selected
time lagged observations (Section \ref{state-space}).  Our theoretical
results identifying tractable problems in earth science require a few
fundamentals from the causal inference literature, so we introduce
causal inference theory, causal graphs, and do calculus for the
unfamiliar reader (Section \ref{sec:causal-graphs-pearls}), and
compare with recent earth science literature on the related field of
causal discovery.

% include this?
% Causal graphs are also a useful general research tool for
% communicating assumptions, whether or not causal inference is actually
% used (Section xx).

\section{Causal graphs and Pearl's do calculus}
\label{sec:causal-graphs-pearls}
% intro causal graphs}

Causal graphs, introduced in \citep{pearl1995causal}, are directed
acyclic graphs (DAGs) that encode our assumptions about the causal
dependencies of a system. One draws directed edges (e.g. arrows)
from variables that are causes to effects. To illustrate, in a
simplified toy example examing clouds, aerosols, and surface solar
radiation, we would draw (Figure \ref{fig:toy}):

\begin{figure}
 % consider \noindent\include... [width=0.75\textwidth]...
 \includegraphics[]{./cloud-aerosol.pdf}\\
 \caption{A relatable toy example to demonstrate basic causal theory.}
 \label{fig:toy}
\end{figure}

\begin{enumerate}
\item An edge from aerosols to clouds because aerosols serve as cloud
condensation nuclei.
\item An edge from aerosols to surface solar radiation, because
aerosols can reflect sunlight back to space and reduce sunlight
at the surface.
\item An edge from clouds to surface solar ratiation, because clouds
also reflect sunlight back to space and can reduce sunlight at
the surface.
\end{enumerate}

Causal graphs encode our assumptions about how the system behaves, and
the nodes and edges that are missing from the graph often represent
strong assumptions about how the system works. For example, in the
cloud-aerosol-sunlight example, clouds also affect aerosols; e.g. by
increasing the likelihood that aerosol will be scoured from the
atmosphere during precipitatation. By not including an edge from cloud
to aerosol, we are making a strong assumption that we are ignoring the
effect of clouds on aerosols. Considering this example is intended to
be pedagocical for introducing causal theory to the readers, we will
continue with the graph as drawn in Figure \ref{fig:toy} (Section
\ref{state-space} explores realistic earth system graphs). Causal
graphs are very useful tools because they can be drawn by any domain
expert with no required knowledge of math or probability, but they
also represent formal mathetmical objects with a very specific
meanings. For example, as a general probabalistic graphical model, the
graph in Figure \ref{fig:toy} represents a specific factorization of
the joint distribution:

\begin{equation}
  p(cloud, aerosol, rad) = p(rad | cloud, aerosol) \, p(cloud |
  aerosol) \,
  p(aerosol)
\end{equation}

Interpretted causally, as in \citet{pearl1995causal}, the directed
edges encode causal meaning (as described above) in addition to a
factorization of the joint. That is, direceted edges go from cause to
effects. /explain how this is different than a probablistic model,
e.g. that there could be many graphs that are consistent with a given
factorizaiton of a joint/

% intro identification}

Given their mathematical meaning, we can use causal graphs theory
\citep[e.g.,][]{pearl1995causal,shpitser2006} to identifiy which
distributions we must estimate from data in order to calcualte a
causal effect of itnerest. This process of identifying the necessary
distributions is formally termed \emph{causal indentification}. If a
causal effect is not identifiable (\emph{un}-identifiable), for example if
calculating a causal effect requires distributions of variables that
we do not observe, then we cannot use causal inference to caluclate a
causal effect, even with an infinite sample of data.

% backdoor path}

A necessary condition for unintifiability is the presenence of an
unblocked backdoor path from cause to effect. Backdoor paths are any
paths going through parents of the cause to the effect. We can block
these paths by selectively observing variables such that no
information passes through them. If we can observe variables such that
backdoor paths are blocked, then we have satisfied the \emph{back-door
  criterion} and we can calulcate unbiased causal effects from
data. Perhaps more intuition can be gained by considering a mutilated
causal graph where we remove all edges from the cause to the cause's
children, and then ask the question, "is the cause independent of the
effect?" in the mutilated graph (Figure {fig:mutilated-toy}). If the
answer is yes, then calcuating our causal effect is identified. If the
answer is no, then we must observe some variables such that the cause
becomes independent of the effect, in the mutilated graph. This
matches intuition: we only want to calculate the co-variability of our
cause and effect due to information travelling from the cause to the
effect along directed causal pathways, not due to the cause's parent
processes inducing covariability in both the cause and effect.

\begin{figure}
 \noindent\includegraphics[]{./mutilated-cloud-aerosol.pdf}\\
 \caption{A mutilation of Figure \ref{fig:toy}, where we have removed
   directed causal paths from the cause (cloud) to the effect (surface
   solar radiation). We can see that there is covariability between
   cloud and surface solar radiation in the data, that is not
   explained by a causal connection between cloud and surface solar
   radiation, but instead is because of induced co-vaiability caused
   by aerosol.}
 \label{fig:mutilated-toy}
\end{figure}

% backdoor path with example}
Perhaps backdoor paths and the backdoor criterion are better
introduced with an example. Returning to out toy example (Figure
\ref{fig:toy}), we will attempt to calcualte the causal effect of
clouds on sunlight. In otherwords, we want to isolate the variability
of sunlight due to the causal link from cloud to sunight. However,
aerosols both effect cloud (edge from aerosol to cloud), and sunlight,
so if we naively calculate a causal effect, for example by just
regressing sunlight on cloud, we would get a biased estimate of the
mean causal effect of cloud on sunlight. This can be shown graphically
if we remove all edges from our cause (cloud) to children of our cause
(in this case sunlight) (Figure \ref{fig:mutilated-toy}). We see that
cloud is not indepednent of aerosol in the mutilated graph. How would
we make cloud and sunlight independent in this mutiliated graph?  In
this case, by observing aerosol. Mathematically, the identification of
the causal effect of cloud and aerosol accoridng to the backdoor
criterion is:

\begin{equation}
  P(sunlight | do(cloud) = c) = \int_{aerosol} p(sunlight| cloud = c,
  aerosol) p(aerosols),
\end{equation}

\emph{introduce ATE?}

where we have implicitly introduced Pearl's "do" calculus
\citep{pearl2009}, which just means we want to calculate the effect of
("doing") an intervention on the cause (in this case cloud) and
setting it to some value of our choosing (in this case \(c\)). In the
case that observations of aerosols are not available, our causal
effect is not identifiable and we cannot use causal inference no
matter how large the sample sizes of clouds and aerosols. This theory
is an elegant tool: without having to touch data or estimate marginal
or conditional distributions, we can determine whether it is possible
to calculate a causal effect of interest. We later use this thoery to
theoretically assess which general problems are tractable in earth
science using causal inference.

Here we focus on the backdoor criterion to block backdoor paths. An
un-blockable backdoor path from the cause to the effect is a necessary
condition for un-intentifiability; however, it is not sufficient
(e.g. there are other identification strategies like the front door
criterion, and instrumental variables). For a complete discussion of
sufficient conditions for un-identifiability we refer you to
\citet{shpitser2006}. However, for the purpose of identifying general
tractable causal inference approaches in earth science (Section
\ref{sec:necess-cond-caus}) , we will focus on blocking backdoor
paths. Given the continously evolving (though time) nature of earth
science systems, and that we only partially observe the state space of
the system (Section \ref{state-space}), these other identification
strategies are not applicable, and an unblockable backdoor path is a
sufficent condition for un-idenfiability in the types of graphs that
are representative of earth science systems (Figure \ref{fig:generic})
/TODO: need to confirm/prove this/.

\subsection{Clarification on terminology and relationship to literature on causal discovery}

% causal discovery = infering causal graph from data}
The term ``causal inference'' has been used to describe two
techniques:

\begin{enumerate}
\item \textbf{Causal effect inference}: Calculating the causal effect
  of some processes on other, given data and assumptions about the
  causal structure of the system.
\item \textbf{Causal structure discovery}: Inferring the causal
  structure (e.g. the causal graph) of the system using data.
\end{enumerate}

This paper focuses on causal inference as (1); calculating causal
effects from data. Inferring the causal structure of the system (2),
is generally much more difficult and requires more
assumptions. However, there are other reasos to focus on (1) rather
than (2) in earth science: often in earth science we know or have a
strong a priori belief about the causal graph of our system. For
example, in the climate system we can identify the state variables and
we know that the state at time \(t\) determnines the state at time
\(t+1\), even if we might not be able to write down the exact
functional form of this state evolution. Therefore, we can write down
a causal graph and do not need to infer graph structrue from data
(Section \ref{sec:necess-cond-caus}).

However, there has been considerable work and effort in applying
causal structure discovery (2) in earth science
\citep[e.g.,][]{ebert-uphoff2012,
  samarasinghe-casuality,runge-causal-timeseries,runge2019inferring}. A
common application and motivation of these efforts is to filter and
ignore causal links in the system through structure discovery, which
is a function of the significance parameters used in the causal
discovery algorithms. Causal inference of effects, as we expolore
here, represents a different approach where we do not rely on
assumptions about the significance of effects, but instead make
explicit assumptions about the causal structure of the system. To
reiterate, given our domain knowledge of the earth science system we
generally have high confidence about the causal structure of the
system, and as we will see (Section \ref{sec:necess-cond-caus}) we can
construct quite general graphs that are faithful to our knowledge
about how dynamical systems evolve. It is possible some links in these
graphs correspond to small effects, and these links would be removed
through causal structure discovery. However, directly interpretting
what constitues a ``negligibly small'' effect calculation presented in
physical units and probabilities, as calculated with causal effect
inference, may be more transparent than intepreting missing links
derived from causal structure discovery significance parameters. In
other words, direct calculation of effects may be more transparent and
interpretable for many readers, relative to a causal graph derived
from significance parameters with more abstract meaning. We hope to
motivate further research effort in causal effect inference to match
recent efforts in causal structure discovery. Ideally causal effect
inference and causal structure discovery will co-evolve as
complementary abstractions for causal intepretation; researchers and
readers can choose the method that suits their assumptions and needs.

\section{Necessary conditions for causal identification in the earth
  systems}
\label{sec:necess-cond-caus}
\subsection{Causal identification when the state space estimable}
\label{state-space}

Earth science systems can be viewed as a stochastic dynamical system
evolving through time according to an underlying state stace. This
offers both advantages and challenges for causal inference. Challenges
involve estimating the state space from observations, while advantages
include the temporal ordering of events; we know that future events
can have no causal effect on the past.

First we consider a generic scenario in a simple temporal system with
no spatial comonent, where we are interested in the causal effect of
$C(t)$ on $E(t+1)$, where $C$ and $E$ are processes within the larger
system's \textit{observed} state $S$. Note that in general we do not
fully observe the state of the system, so there will also be an
unobserved portion of the state space $U$. We denote the portion of
the observed state space excluding $C$ and $E$ as $S'$. $S'(t)$,
$U(t)$, $C(t)$ and $E(t)$ all affect $S'(t+1)$, $U(t+1)$, $C(t+1)$,
and $E(t)$ (Figure \ref{fig:generic}). In order to calculate the
causal effect of $C(t)$ on $E(t+1)$, we must block all back door paths
from $C(t)$ to $E(t+1)$. In this generic scenario, this involves
conditioning on $S'(t)$, $U(t)$, and $E(t)$:

\begin{figure}
  \includegraphics[]{./generic-graph.pdf}
  \caption{A generic graph of the earth system sequence, limited to a
    3 time sequence subset of the infinite sequence. $S$ and $U$ refer
    to the observed and unobserved portions of the state space,
    respectively.  At time steps $t$ and $t+1$ the observedn portion
    of the state space is broken up into three events: $C$, $E$, and
    $S'(t)$, which is all parts of $S$ excluding $C$ and $E$. $C$ is a
    generic portion of the state space we are intested in as a cause, and $E$ is
    a generic portion of the state space we are interested in as an effect.}
  \label{fig:generic}
\end{figure}

\begin{equation}
  \label{naive}
  P(E(t+1)| do(C(t))) = \int_{S'(t)} \int_{U(t)} \int_{E(t)}  p(E(t+1) | C(t), S'(t),
  U(t), E(t)) \; p(E(t)) \; p(U(t)) \; p(S'(t))
\end{equation}

However, we do not observe $U(t)$, and cannot block this path. We may,
however, be able to estimate $U(t)$ by leveraging Takens's theorem and
using lagged observations of the partial state space, going back in
time (e.g. $S(t-1)$, $S(t-2)$, etc.). In this application, this
approach presents problems. The reconstruction will estimate $S$
rather than $S'$, which includes $C(t)$ because $C$ is a part of the
state space. Fortunately, we can also block backdoor paths by
conditioning on $S(t-1)$ and $U(t-1)$. If we reconstruct $U(t-1)$
using lagged observations of $S$ before $t-1$, we can block backdoor
paths and calculate a causal effect. So, the more tractable approach
to blocking backdoor paths in the earth system, where we only ever
partially observe the state space, is:

\begin{equation}
  \label{ce}
  P(E(t+1)| do(C(t))) = \int_{S(t-1)} \int_{U(t-1)} p(E(t+1) | C(t), S(t-1),
  U(t-1)) \; p(U(t-1)) \; p(S(t-1)),
\end{equation}

where $U(t-1)$ is statistically reconstructed using lagged
observations of the state space $S(t')$, where
$t' \in \{-\infty, t-2\}$ (justified by Takens's theorem).

To extend this result on a purely temporal dyanmical system to a
spatio-temporal system, we use the idea of lightcones
\citep{PhysRevLett.84.1890,
  montanez2015licors,doi:10.1063/1.5021130}. Lightcones leverage our
general knowledge or assumptions about the speed with with the
influence of events can propagate through the system. For a general
physical system, an upper limit on the propagation speed is the speed
of light. For the atmosphere, a useful upper limit is the speed of
sound. However, in many scenarios (e.g. if we do not care about the
effect of soundwaves), we can use a more reasonalbe speed of
propagation like the advection (wind) speed, or the phase/group speed
of gravity waves. We can use the propagation speed of the system to
analyze what portions of the state space can or cannot affect an event
at a given time and spatial extent. The past lightcones are the set of
all events in a spatiotemporal system that can \textit{affect} a
single event at a given location and time. The future lightcones are
the set of all events in a sptiotemporal system that are
\textit{affected by} an event at a given location and time. Formally,
the past lightcone ($L^-$) is the set fo all events defined by:

\begin{equation}
  L^-(r,t) \equiv \left\{ X_{t'}^{r'} \, : \, t' \leq t \text{ and }
      ||r'-r|| \leq c(t-t')\right\},
\end{equation},

where $c$ is the propagation spped. The future lightcone is defined
similarly:

\begin{equation}
  L^+(r,t) \equiv \left\{ X_{t'}^{r'} \, : \, t' \geq t \text{ and }
      ||r'-r|| \leq c(t'-t) \right\}
\end{equation}.

We can use the past lightcones of our cause and out effect to
determine the spatio-temporal locations we must use to estimate
$S(t-1)$ and block backdoor paths.

An example will aid visualizing lightcones, and their consequences for
causal inference. Consider a scenario where half-hourly observations
are available, we assume the propagation speed is $25 m/s$, and that
we are interested in the affect of $C(x=0 m, y=0 m, t=0 hr)$ on
$E(x=0, y=0, t=0.5 hr)$. In this case, we would need to estimate
$S(t=-0.5 hr)$, over all locations that could affect \emph{both}
$C(0,0,0)$ and $E(0,0,0.5)$.  For a propagation speed of $25 m/s$,
this requires esimating the state over a circular area centered at
$x=0$, $y=0$, and with a radius $25 \, m/s \cdot 1800 \, s = 45
km$. However, becuse we only partially observe the state space, we
must use lagged observations going back in time, inlcluding all
observations of events within the past lightcone of the system. For
example, at $-1$ hours our circular radius would be 90 km, at $-1.5$
hr: 135 km, and at $-2$ hr: 180 km. As one can observe, the problem
because stastically and computationally harder the more lags are
required to estimate are state space. If the number of lags in
inversly proportional to the obsevational coverage of the state space,
this motivates more observations of the earth system!

/where to put stuff about software I wrote?/

In practice we do not know how many lagged observations of the state
space we need to reconstruct it at time $t-1$. One possible approach
is to use the data for guidance: we will have fully reconstructed the
state space when the addition of more time lags does not imporve our
prediciton of our effect at time \(t+1\) (or cause at time
$t$). /think about this prediction of cause and effect, in terms of
``predictively important variables may have limited causal meanings''
in my notes/. goal should be


\subsection{Applications at the human-earth system interface: when the
cause is approximately independent of the system}
\label{human}

One particularly tractable approach that avoids estimating the full
state space of our system at a given time, is when our causes of
interest are independent from the evolution of the state space. While
the earth system certainly affects all objects living on earth, in
some cases this may be a reasonable assumption. For example, it is
tragic but unfortunately recent human history has demonstrated that
our global actions are relatively independent of the climate
state. That is, we have failed to reduce green house gas emissions
even as global temperature increased. When global green house gas
emissions have risen less sharply, it is usually due to global
economic recession (cite xx, 2008). In this case it appears that the
many global social, policital and economic factors are primary causes
of global green house gas emission, and while the climate system may
effect this forces, the historical evedence suggests that an
assumption that global green house gas emission is independent fo the
climate may be reasonable, at least for the present and near future
\emph{discuss data problem - we know don't observe the future of green
  house gas emissions}.

Generally this logic applies to many scenarios on the
"human-climate" interface. For example, land-use land-cover change
can be viewed in many scnearios independent of the climate system,
as in urban centers where urban planning is relatively independent
fo recent climate history. We encode a general graph of the
assumption that causes are independent of the state space,
reflecting our general suggestion that the human-climate interface
may be a particularly common example of this scenario.

\section{Discussion}

\begin{itemize}
\item Causal inference from data is a new tool that can traditional
  scientific methods, incluidng numerical and real world
  experimentation. In earth science, numerical models rely on
  approximations that deviate their behavior from reality, and real
  world experimentation may be intractable or unethical. Causal
  inference is a third tool to calculate the effects of physical
  processes that has a differerent set of advantages and
  disadvantages, and is a powerful complement to numerical dn real
  world experimentation.
\item The characteristics of earth science systems offer advantages
  for causal ifnerence: the temporal ordering of events and bounds on
  the propagations speed of information in the system allows us to
  filter which events that can affect a process, and identify the
  necessary portions of the state space we must condition on to block
  backdoor paths.
\item Applying causal inference to earth science systems also presents
  challenges: we only ever partially observe the state space of the
  system, so we must reconstruct the full state space using time
  lagged observations. This increases the computational and
  statistical complexity of the problem, and can make some application
  untractable.
\end{itemize}

/discuss transportability?/

/discuss relationship of these ideas to causal discovery (critique
causal discovery)?/

/discuss issues with data available and observed range
(e.g. generalization for cliamte research)?/

\bibliography{references.bib}

\end{document}