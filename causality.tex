\documentclass[12pt]{article}
% fonts
\usepackage[scaled=0.92]{helvet}   % set Helvetica as the sans-serif font
\renewcommand{\rmdefault}{ptm}     % set Times as the default text font

% db/ not mandatory, but i recommend you use mtpro for math fonts.
% there is a free version called mtprolite.

% \usepackage[amssymbols,subscriptcorrection,slantedGreek,nofontinfo]{mtpro2}

\usepackage[T1]{fontenc}
\usepackage{amsmath}
\usepackage{amsfonts}
\usepackage{pgf}
% page numbers
\usepackage{fancyhdr}
\fancypagestyle{newstyle}{
  \fancyhf{} % clear all header and footer fields
  \fancyfoot[R]{\vspace{0.1in} \small \thepage}
  \renewcommand{\headrulewidth}{0pt}
  \renewcommand{\footrulewidth}{0pt}}
\pagestyle{newstyle}

% geometry of the page
\usepackage[top=1in,
bottom=1in,
left=1in,
right=1in]{geometry}

% paragraph spacing
\setlength{\parindent}{0pt}
\setlength{\parskip}{2ex plus 0.4ex minus 0.2ex}

% useful packages
\usepackage{natbib}
\bibliographystyle{plainnat}
\usepackage{epsfig}
\usepackage{url}
\usepackage{bm}

\usepackage{titlesec}
\titleformat*{\section}{\normalsize\bfseries}
\titleformat*{\subsection}{\normalsize\bfseries}

\usepackage{enumitem}
\setlist{nolistsep}


\begin{document}

\title{Causal inference in earth science}

\author{Adam Massmann\thanks{Corresponding author: akm2203@columbia.edu}}

\maketitle

\begin{abstract}
  Research on the inference of causal effects from data is rapidly
  intensifying, and earth system scientists are applying new causal
  inference methods to a variety of problems. I examine the
  theoretical justification for applying causal inference methods to
  climate science. Because we generally do not fully observe the state
  space of the problem, the assumptions behind generic causal
  inference methods break down. However, by focusing on problems at
  the human-environment interface (e.g. anthropogenic greenhouse gas
  emissions and urban heat island effect), we may be able to
  successfully apply causal inference methods to earth science and
  calculate the causal effects of humans on the environment.

\end{abstract}

\section{Introduction}

% usefullness causal inference e.s.

Causal inference is the estimation of effects (e.g. temperature)
in response to changes in causes (e.g. green house gas
concentration) using observational data from a system. Causal
inference is an alternative path towards scientific discovery,
which traditionally relies on experimentation (e.g. numerical,
lab, or real world) to determine the effect of varying causes. In
earth sciences in particular it may be impossible or unethical to
execute real world experiments, and numerical experiments may rely
on approximations that bias causal estimates relative to the real
world. Given these challenges in earth science, causal inference
from data is a tantilizing tool that allows for robust
estimatation of causal effects that may be otherwise impossible
using traditoinal tools.


% challenges of ci in e.s.}

However, causal inference from data presents its own challenges. A
necessary conditon for calculating a causal effect is that the
causal effect must be identifiable from the set of observable
variables. Whether or not a causal effect is identifiable is
determined from our assumptions about how a system behaves and
which variables we can obtain observations for. If an effect is
determined to be unidentifable, then it is impossible to calculate
the causal effect of interest, even with an infitinte sample of
data. Specific to problems in earth science, the identification of
causal effects is challenging given two fundamental properties of
many generic earth science problems: the system evolves through
time according to an underlying dynamical system, and we only ever
partially observe the state space of that dynamical system. Using
graph theory developed in \citep{pearl-1995} we determine the
generic problems causal inference is justified for, and the
necesarry assumptions for applyign causal inference. There is the
potential for applying causal inference in two scenarios: (1)
under an assumption that the cause is independent of the (cliamte)
system's state space (e.g. at the human-climate interface,
anthropogenic green house gas emission, land use change) (Secion
xx); and (2) under certain conditions where we can estimate the
state space of the system from using carefully selected time
lagged observations (Section xx).  \emph{discuss relationship to
boundary condition experiments?} Our theoretical results
identifying tractable problems in earth science require a few
fundamentals from the causal inference literature, so we introduce
causal inference theory, causal graphs, and do calculus for the
unfamiliar reader (Section xx), and compare with recent earth
science literature on the related field of causal discovery.

% include this?
% Causal graphs are also a useful general research tool for
% communicating assumptions, whether or not causal inference is actually
% used (Section xx).

\section{Causal graphs and Pearl's do calculus}

% intro causal graphs}

Causal graphs, introduced in \citep{pearl-1995}, are directed
acyclic graphs (DAGs) that encode our assumptions about the causal
dependencies of a system. One draws directed edges (e.g. arrows)
from variables that are causes to effects. To illustrate, in a
simplified toy example examing clouds, aerosols, and surface solar
radiation, we would draw (Figure xx):

\begin{enumerate}
\item An edge from aerosols to clouds because aerosols serve as cloud
condensation nuclei.
\item An edge from aerosols to surface solar radiation, because
aerosols can reflect sunlight back to space and reduce sunlight
at the surface.
\item An edge from clouds to surface solar ratiation, because clouds
also reflect sunlight back to space and can reduce sunlight at
the surface.
\end{enumerate}

Causal graphs encode our assumptions about how the system behaves,
and the nodes and edges that are missing from the graph often
represent strong assumptions about how the system works. For
example, in the cloud-aerosol-sunlight example, clouds also affect
aerosols by increasing the likelihood that aerosol will be scoured
from the atmosphere during precipitatation. By not including an
edge from cloud to aerosol, we are making a strong assumption that
we are ignoring this effect of clouds on aerosols. Cosidering this
example is intended to be pedagocical for introducing causal theory
to the readers, we will continue with the graph as drawn in Figure
xx. Causal graphs are very useful tools because they can be drawn
by any domain expert with no required knowledge of math or
probability, but they also represent formal mathetmical objects
with very sepcific meanings (discuss factorization of joint?).

% intro identification}

Given their mathematical meaning, we can use causal graphs to
identifiy which distributions we must estimate from data in order
to calcualte a causal effect of itnerest. This process of
identifying the necessary distributions is formally termed
\emph{causal indentification}. If a causal effect is not
identifiable (/un/identifiable), for example if
calculating a causal effect requires distributions of variables
that we do not observe, then we cannot use causal inference to
caluclate a causal effect, even with an infinite sample of data.

% backdoor path}

An early, well studied and relevant (to climate science)
sufficient condition for identificatication is called the
\emph{backdoor criterion}. The backdoor criterion states that if we can
observe variables such that the backdoor paths from cause to
effect are blocked, then a causal effect is identified. Backdoor
paths are any paths going through parents of the cause, to the
effect. We can block these paths by selectively observing
variables such that no information passes through them. Perhaps
more intuition can be gained by considering a mutilated causal
graph where we remove all edges from the cause to the cause's
children, and then ask the question, "is the cause independent of
the effect?" in the mutilated graph. If the answer is yes, then
calcuating our causal effect is identified. If the answer is no,
then we must observe some variables such that the cause becomes
independent of the effect, in the mutilated graph. This matches
intuition: we only want to calculate the co-variability of our
cause and effect due to information travelling from the cause to
the effect along directed causal pathways, not due to the cause's
parent processes inducing covariability in both the cause and
effect. In other words, in the absence of backdoor paths,
correltion is causation.

% backdoor path with example}

Perhaps backdoor paths and the backdoor criterion are better
introduced with an example. Returning to out toy example Figure
xx, we will attempt to calcualte the causal effect of clouds on
sunlight. In otherwords, we want to isolate the variability of
sunlight due to the causal link from cloud to sunight. However,
aerosols both effect cloud (edge from aerosol to cloud), and
sunlight, so if we naively calculate a causal effect, for example
by just regressing sunlight on cloud, we would get a biased
estimate of the mean causal effect of cloud on sunlight. This can
be shown graphically if we remove all edges from our cause (cloud)
to children of our cause (in this case sunlight) (Figure xx). We
see that cloud is not indepednent of aerosol in the mutilated
graph. How would we make cloud and sunlight independent in this
mutiliated graph? In this case, by observing
aerosol. Mathematically, the identification of the causal effect
of cloud and aerosol accoridng to the backdoor criterion is:

\begin{equation}
  P(sunlight | do(cloud) = c) = \int_{aerosol} p(sunlight| cloud = c,
  aerosol) p(aerosol)),
\end{equation}

\emph{introduce ATE?}

where we have implicitly introduced Pearl's "do" calculus, which
just means we want to calculate the effect of ("doing") an
intervention on the cause (in this case cloud) and setting it to
some value of our choosing (in this cas \(c\)). In the case that
observations of aerosols are not available, our causal effect is
not identifiable and we cannot use causal inference no matter how
large the sampel sizes of clouds and aerosols. This theory is an
elegant tool: without having to touch data or estimate marginal or
conditional distributions, we can determine whether it is possible
to calculate a causal effect of interest. We later use this thoery
to theoretically assess which general problems are tractable in
earth science using causal inference.

% other identification strategies focusing on backdoor path}

Here we focus on the backdoor criterion. However, other
identification stregies exist, primarily the front-door criterion
and instrument variables. Given that the front-door criterion can
be reformulated as a specific case of the back-door criterion
(cite XX?), and that applying instrument variables to a system
with an evolving state space is challenging (cite XX? - also
discuss how in practice intrument variables can introduce
biases? - see sSHalizi). However please consult citeXX and cite
XXX for more information on the front-door criterion and
instrument variables.

\subsection{Clarification on terminology and relationship to literature on causal discovery}

% causal discovery = infering causal graph from data}

In this paper we discuss causal inference, and define it as
estimating the induced effect of intervening on a cause (setting
it or changing it to some value). However "causal inference" has
also been used previously to describe the process of inferring the
graph strucutre of a causal graph from data. To be consisstent
with previous literature in the causal community, we call this
technique of of infering the graph structure from data \emph{causal
structure discovery}, whiel calculating causal effects given a
graph and data is \emph{causal inference}. There has been cosiderable
recent focus on \emph{causal structure discovery} in earth science, so
it is worthwhile to discuss the relationship between causal
inference and causal discovery and how it relates to earth science.

% often times we know causal graph (can write down equations)}

Often in earth science we know or have a strong a priori belief
about the causal graph of our system. For example, in the climate
system we can identify the state variables and we know that the
state at time \(t\) determnines the state at time \(t+1\), even if we
might not be able to write down the exact functional form of this
state evolution. Therefor, we can write down a causal graph and do
not need to infer graph structrue from data.  So, why the itnerest
in causal discovery for earth science? One possible explanation is
that causal discovery may be useful to remove edges that have a
small or negligble effect. Some portions of the climate state at
time \(t\) may have a negligble effect on other portions of the
climate state at time \(t+1\), and causal discovery could eliminate
edges between these portions of the state space.  Additionally,
while we can usually derive a causal graph of an earth science
system using expert knowledge for physical variables, we may not
be able to do so for transformed or reduced-dimension derived
quantities. Causal discovery methods can be applied to identify
causal dependencies between between these derived variables.

% resulting causal graph will be function of significance paramters}

However, causal discovery algorithms rely on significance
parameters used to determine conditoinal independence between
variables. Because causal discovery algorhtms can produce
potentially arbitraty results as a function of hyperparameters
(significance levels) of the method, it may be methodloically
safer to build a causal graph using expert knowledge, even if this
limits us to scenarios where we know the causal structure (in
physical sciences this may not be so limiting). It is still
possible to test that the causal sgraph is consistent with
observations using conditional independence tests, so we can still
verify some of the assumptions in a causal graph.

% many pitfalls of causal discovery same as for causal inference}

Finally, many pitfalls of causal discovery are similiar for causal
inference. In particular, applying causal discovery algorithms on
a system with only partial overservations of the presents
significant unresolved problems and challenges (cite XX). Given
that we often know the causal structure of our system through well
studied physics and dynamics, we believe there is a lot of
potential for using causal inference. Given this motivation we
focus the rest of our discussion on the problems for which causal
inference is tractable in earth science.

 % put in this section? think important to discuss regression and
 % problems of conditioning on variables that are in the causal path.
 % ** Causal graphs: a tool for communicating assumptions and avoiding flawed experiments
 % *** A tool for communicating assumptions
 % *** Avoiding flawed experiment designs

\subsection{Applications in earth science: pitfalls and justified
  approaches for generic scena
  rios}

% reorganize as: what does it take to block backdoor paths. okay, what
% about the fact that we partially observe state space?

% also, start with discussing internal to the system, then bring up how
% we might be able to get away with assuming soem pieces are external to
% the system

% Earth science as a dynamical system with partial observations of the state space}

Earth science systems can be viewed as a dynamical system evolving
through time according to an underlying state stace. However, in
the earth system we only every partially observe this state
space. When encoded in a generical causal graph (Figure XX), we
can see that there will always be open backdoor paths from a cause
in any part of the state sapce, to an effect in andy part of the
state space, through the unobserved portions of the state
space. So, in this general earth science scenario, causal
inference is unidentifiable, and thereofor impossible even with an
infitie amount of data. However, are there some additional
assumptions that are consistent with common and relevant scenarios
under which causal inference is tractable for a partially observed
state space?

% Tractable approach: causes are independent if the state space}

One tractable approach is when our causes of interest are
independent from the evolution of the state space. While the earth
system certainly affects all objects living on earth, in some
cases this may be a reasonable assumption. For example, it is
tragic but unfortunately recent human history has demonstrated
that our global actions are relatively independent of the climate
state. That is, we have failed to reduce green house gas emissions
even as global temperature increased. When global green house gas
emissions have risen less sharply, it is usually due to global
economic recession (cite xx, 2008). In this case it appears that
the many global social, policital and economic factors are primary
causes of global green house gas emission, and while the climate
system may effect this forces, the historical evedence suggests
that an assumption that global green house gas emission is
independent fo the climate may be reasonable, at least for the
present and near future \emph{discuss data problem - we know don't
observe the future of green house gas emissions}.

Generally this logic applies to many scenarios on the
"human-climate" interface. For example, land-use land-cover change
can be viewed in many scnearios independent of the climate system,
as in urban centers where urban planning is relatively independent
fo recent climate history. We encode a general graph of the
assumption that causes are independent of the state space,
reflecting our general suggestion that the human-climate interface
may be a particularly common example of this scenario.

% Tractable approach: thoughtfull estimation of the state space with Takens' theorem}

A second causal scenario, more applicable for studying behavior
within the earth system, is when we can estimate the state space
using statistical tools and Takens' theorem. In this case, if we
are interested in calculating a an effect, we can block backdoor
paths from the cause to the effect with the unobserved portion of
the state space. However, care must be taken when estimating our
state space. The observations we use to estimate our state space
must:

\begin{enumerate}
\item Not lie on the directed causal path from the cause to the
effect.

\item Must represent all portions of the statespace that affect both
the cause and the effect.
\end{enumerate}

Fortunately, in earth system obsevations we know the temporal
ordering of events, and we also generally know how quickly
information at a given location can move to affect a process at
another location and time. For example in for a general physics
scenario information travels at the speed of light, but for the
atmosphere a very conservative propagation speed would be the
speed of sound. If we wish to ignore the effect of sound waves, we
could alternatively use a less conservative propagtion speed such
as a maximum bound on gravity wave phase/group speed or advection
(wind) speed. In either case, if we know the timing and location
of our cause, and the timing and location of our effect, then we
know the spatial extent over which we need to estimate our state
space so that backdoor paths form our cause and our effect are
blocked. An example may help. Say we are interested in the effect
of soil moisture (\(SM\)) at time (\(t=0\) hr) and location (\(x=0\) m,
\(y=0\) m) on average evapotranspiration (ET) over the next hour
(from \(t=0\) \emph{introduce "lightcone" idea from shalizi?} hr to \(t=1\)
hr and at the location \(x=0\) m, \(y = 0\) m. Soil moisture can
effect the state, so we do not want to condition on any
oveservations after time \(t=0\). But, if we know the relevant state
space at time \(t=0\) which affects both soil moisture and ET, we
can block back door paths by conditionin gon that state space.
However, as previously recognized, we never fully observe the
state space of the system, so we must use Takens' theorem and
statistical tools to "fill in" unobserved portions of the state
space with lagged observations. This makes the problem harder: we
must use observations over a larger temporal and spatial extent
than if we fully observed the state space, in order to estimate
the state space and block backdoor paths. In the soil moisture
example, let us say that with the observations abailable, 3
instances of hourly observations (e.g. (\(t-2\), \(t-1\), and \(t\)) is
enough to reconstruct the state space at time \(t\).  However, if we
take a coservative approach and assume that the maximum speed of
propagation through the atomosphere is the speed of sound
(\(\approx 340 m s^{-1}\)), then we would need to use observations
over a radius of 2248 km radius at time \(t-2\), a 1224 km at time
\(t-1\), and at \((x=0, y=0)\) at time \(t\). This is a much bigger
(statistical) problem. Fotunately, much historical work in
atmospheric science has defined relevant time, space, and speed
scales over which perturbations propagate vertically and
horizonatally (e.g. rossby deformation radius, burger's number,
group/phase velocity of gravity waves, etc.). We can use this work
to define less conservative, but still resonable assumptions about
the area over which we must calculate the state space which makes
the problem more computationally tractable.  For example, if we
assume taht the speed or propagation is \(50 m s^{-1}\), then we
would need to use observations over a radius of 360 km at time
\(t-2\), and 180 km at time \(t-1\). Also, here we are focusing on
just a simple radius based approach with a fixed propagation speed
for ease of interpretation, but the size of the problem could
perhaps be reduced by looking more thoughtfully at the data, and
including advecion directions and sppeds to reduce the area from
the naive radius-based approach. /this could be a little more
complicated and relies on an assumption that air does not
circulate back to a location e.g. at time t+1 (end of average),
soil moisture is affected by a radius of 180 km, but there are no
gaurantees that that area did not affect soil moisture at time t,
and also \textbf{is} included in the state at time t, x=0, y=0./ \emph{aslo
think about how to handle SM with large "memory"}. \emph{what about
plants- healtha nd adaptation (unobserved)?}

So far we have assumed that we know how many lagged samples of
observations are necessary to reconstruct the observed state
space, but in practice we do not know how much we need. One
possible approach is to use the data for guidance: we will have
fully reconstructed the state space when the addition of
more time lags does not imporve our prediciton of our effect at
time \(t+1\). Returning to the soil moisture example, we could
determine the number of time lags by calculating predictions base
on different set of lagged observations (e.g. \({t}\), \([t, t-1]\),
\({t, t-1, t-2}\), etc.). We select the number of time lagged
obsercations necessary to reconstruct the state space relcant for
soil moisutre as the set that maximizes the predicitive
skill. This approach also favors our simple method of assuming a
propagation speed and a radius (lightcone?) of influence, ad any
advection direction and speeds vary though time, so we can not
determine the necessary number of lags using a timeseries of
obsercations, as detailed above. \emph{review and tie back to shalizi
light cone?} \emph{think about and include importance of including
effect in regression tests}.

\subsection{Discussion}

discuss transportability?

discuss relationship of these ideas to causal discovery (critique
causal discovery)?

discuss issues with data available and observed range
(e.g. generalization for cliamte research)?

\end{document}