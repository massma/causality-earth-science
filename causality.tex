\documentclass[12pt]{article}
% fonts
\usepackage[scaled=0.92]{helvet}   % set Helvetica as the sans-serif font
\renewcommand{\rmdefault}{ptm}     % set Times as the default text font

% db/ not mandatory, but i recommend you use mtpro for math fonts.
% there is a free version called mtprolite.

% \usepackage[amssymbols,subscriptcorrection,slantedGreek,nofontinfo]{mtpro2}

\usepackage[T1]{fontenc}
\usepackage{amsmath}
\usepackage{amsfonts}
\usepackage{pgf}
% page numbers
\usepackage{fancyhdr}
\fancypagestyle{newstyle}{
  \fancyhf{} % clear all header and footer fields
  \fancyfoot[R]{\vspace{0.1in} \small \thepage}
  \renewcommand{\headrulewidth}{0pt}
  \renewcommand{\footrulewidth}{0pt}}
\pagestyle{newstyle}

% geometry of the page
\usepackage[top=1in,
bottom=1in,
left=1in,
right=1in]{geometry}

% paragraph spacing
\setlength{\parindent}{0pt}
\setlength{\parskip}{2ex plus 0.4ex minus 0.2ex}

% useful packages
\usepackage{natbib}
\bibliographystyle{plainnat}
\usepackage{epsfig}
\usepackage{url}
\usepackage{bm}

\usepackage{titlesec}
\titleformat*{\section}{\normalsize\bfseries}
\titleformat*{\subsection}{\normalsize\bfseries}

\usepackage{enumitem}
\setlist{nolistsep}


\begin{document}

\title{Causal inference in earth science}

\author{Adam Massmann\thanks{Corresponding author: akm2203@columbia.edu}}

\maketitle

\section{Introduction}

Controlled experimentation is the traditional path to scientific
discovery. A scientist designs controlled experiments where parameters
are systematically varied to test hypotheses about a system's
nature. Climate and earth system scientists primarily use numerical
models for experimentation \citep[e.g.,][]{eyring-cmip6-2016}, but
real world experiment is also used to a lesser degree
\citep[e.g.,][]{ainsworth-face-2005}. However, it may be logisically
impossible or unethical to execute many real world experiments, and
numerical experiments may rely on approximations that bias
experimental results realtive to the real world
\citep[e.g.,][]{kim-cmip5,stillmann-cmip5-extremes}. Given the
limitations of real world and numerical experiments in earth science,
researchers require additional tools for sceintific discovery.
\citet{pearl-1994-do-calculus} introuduced just one such tool: the
\textit{do-}calculus for calculating the causal effects of
experimenatal interventions based on passive observations of a system
and assumptions about causal dependencies in the system. In other
words, given observational data about a system one can use
\textit{do}-calculus to calulate an experimental effect without
conducting any experiments. This is a powerful tool for earth
scientists that complements numerical experimentation due to a
completely orthogonal set of strengths, weaknesses, and
assumptions. When numerical modelling assumptions break down, causal
inference may be a tractable path towards scientific discovery.

However, causal inference from data presents its own challenges; the
causal effect must be identifiable from assumptions about the causal
dependencies of variables in the system, and the available
observations. If a causal effect is determined to be unidentifiable,
then it is impossible to calculate the causal effect of interest, even
with an infinite sample of data \citep[][]{shpitser2006}. Specific to
problems in earth science, the identification of causal effects is
challenging given two fundamental properties of many generic earth
science problems: the system evolves through time according to an
underlying dynamical system, and we only ever partially observe the
state space of that dynamical system \citep{majda-state}. Using graph
theory developed in \citep{pearl1995causal} we determine the generic
problems for which causal inference is justified, and the necessary
assumptions for applying causal inference. There is the potential for
applying causal inference in two scenarios: (1) under an assumption
that the cause is independent of the system's state space (e.g. at the
human-climate interface, anthropogenic green house gas emission, land
use change) (Section \ref{human}); or (2) under certain conditions
where we can estimate the state space of the system from time lagged
observations (Section \ref{state-space}).  Our theoretical results
identifying tractable problems in earth science require a few
fundamentals from the causal inference literature, so we introduce
causal inference theory, causal graphs, and \citet{pearl2009}'s
\textit{do-}calculus for the unfamiliar reader (Section
\ref{sec:causal-graphs-pearls}), and compare with recent earth science
literature on the related field of causal discovery (Section
\ref{sec:discovery}). Causal effects, due to their underlying logic
and clear interpretation, are an exciting tool for scientific
discovery and understanding \citep{hannart-da,naveau-2020}. This
manuscript establishes the conditions required for the proper
application of causal inference in earth science.

\section{Causal graphs and Pearl's do calculus}
\label{sec:causal-graphs-pearls}
% intro causal graphs}

Causal graphs, introduced in \citep{pearl1995causal}, are directed
acyclic graphs (DAGs) that encode our assumptions about the causal
dependencies of a system. One draws directed edges (e.g. arrows) from
variables that are causes to effects. We will demonstrate causal
graphs with a simplified toy example examining clouds, aerosols, and
surface solar radiation/sunlight (Figure \ref{fig:toy}). The causal
graph consists of:

\begin{figure}
  % consider \noindent\include... [width=0.75\textwidth]...
  \includegraphics[height=0.4\textheight]{./cloud-aerosol.pdf}\\
  \caption{A relatable toy example to demonstrate basic causal theory.}
  \label{fig:toy}
\end{figure}

\begin{enumerate}
\item An edge from aerosols to clouds because aerosols serve as cloud
  condensation nuclei and affect the probability of water vapor
  conversion to cloud.
\item An edge from aerosols to surface solar radiation, because
  aerosols can reflect sunlight back to space and reduce sunlight
  at the surface.
\item An edge from clouds to sunlight, because clouds also reflect
  sunlight back to space and can reduce sunlight at the surface.
\end{enumerate}

Causal graphs encode our assumptions about how the system behaves, and
the nodes and edges that are \textit{missing} from the graph often
represent strong assumptions. For example, in the
cloud-aerosol-sunlight example, clouds also affect aerosols; e.g. by
increasing the likelihood that aerosol will be scoured from the
atmosphere during precipitation. By not including an edge from cloud
to aerosol, we are making a strong assumption that we are ignoring the
effect of clouds on aerosols. Considering this example is intended to
be pedagogical for introducing causal theory to the readers, we will
continue with the graph as drawn in Figure \ref{fig:toy} (Section
\ref{state-space} explores realistic earth system graphs). Causal
graphs are very useful tools because they can be drawn by any domain
expert with no required knowledge of math or probability, but they
also represent formal mathematical objects with a very specific
meanings. For example, as a general probabilistic graphical model, the
graph in Figure \ref{fig:toy} represents a specific factorization of
the joint distribution:

\begin{equation}
  P(A, C, S) = P(S \, | \,C, A) \, P(C \, | \, A) \, P(A),
\end{equation}

where $A$ represents aerosol, $C$ represents cloud, and $S$
representes surface sunlight/solar radiation, Interpreted causally, as
in \citet{pearl1995causal} and this manuscript, the directed edges
encode causal dependencies in addition to a factorization of the
joint. More specifically, the graph represents the assumed Structural
Causal Model \citep[SCM,][]{pearl2009}, which is the set of functions
that determines how variables are generated by their causes. Each
variable in the graph is determined by a function with inputs
corresponding to inward edges of the variable of interest, as well as
randomness due to the variables not included in the causal graph
(``exogenous variables'').  In the cloud-aerosol-radiation example the
SCM is:

\begin{equation}
  SCM =
  \begin{cases}
    f_{aerosol} &: U_{aerosol} \\
    f_{cloud} &: aerosol, U_{cloud}  \\
    f_{sunlight} &: aerosol, cloud, U_{sunlight}
  \end{cases}
  \label{eq:1}
\end{equation}

The SCM asserts that each variable is generated from a deterministic
function. The randomness is due to exogonous variables ($U$) that
represent all the factors not represented explicitly in the SCM and
corrresponding causal graph. For example, sources of aerosol
variability not considered include anthropogenic aerosol emission, the
biosphere, fires, volcanoes, etc. For cloud, this includes synoptic
forcing, moisture content, etc. For radiation, this includes
variability of top of atmosphere radiation, etc. For the causal graph,
as drawn, we assume that all of these sources of randomness are
idnependent of each other. That is, $U_{aerosol}$, $U_{cloud}$, and
$U_{radiation}$ are independent of each other. The $U$ variables are
powerful tools: by positing an SCM and causal graph, we are not
stating that the variables in the graph are the only processes in the
system. Instead, we are stating that all other processes not included
in the graph induce variations in the causal variables that are
uncorrelated with each other. In some scenarios, this assumption may
be unreasonable and we would need to encode correlations between $U$
terms; that is, co-variability induced between variables due to
exogenous processes. This can be encoded in the graph as a bi-directed
dashed edge between the two variables with shared exogenous
factors. For example, if aerosol in Figure \ref{fig:toy} were istead
considered to be exogenous, we would represen the graph instead as in
Figure \ref{fig:bi-directed}, and the corresponding SCM would be:

\begin{figure}
  \includegraphics[]{bidirected.pdf}
  \caption{A graph equivalent to Figure \ref{fig:toy}, but under an
    assumption that aerosol, a source of variability in both cloud and
    surface radation, is exogenous. In this case, there is a dependency
    between the randomness of cloud and radiation.}
  \label{fig:bi-directed}
\end{figure}

\begin{equation}
  SCM =
  \begin{cases}
    f_{cloud} &: U_{cloud,radiation}  \\
    f_{radiation} &: cloud, U_{cloud,radiation}
  \end{cases},
  \label{eq:2}
\end{equation}

where cloud and radiation now share an exogenous term. In this paper
we do not include the bi-directed dashed edge notation, because in
earth science we can usually identify the relevant variables and their
dependencies, which is sufficient infrmation to include them
expilcitly in the graph as nodes, even if we do not know the
functional form of their dependencies or cannot observe them.

\textit{/include this discussion barienboim's notation (which I do not
  like) of including and representing exogoneous variation that
  affects multiple variables with a dashed line? not as applicable for
  earth science, where I *think* we have a general idea of the main
  players (e.g. state), and how they relate to each other. I prefer to
  represent these instead as unobservable $V$ than $U$. However, show
  it in this example by just saying, if we did not include aerosol as
  an endogenous variable we would just represent that as a bi-directed
  dashed arrow from $V$ to $U$./}

We can apply causal graph theory
\citep[e.g.,][]{pearl1995causal,shpitser2006} to the assumptions
encoded in our causal graph to identify which distributions we must
estimate from data in order to calculate a causal effect of
interest. This process of identifying the necessary distributions is
formally termed \emph{causal identification}. If a causal effect is
not identifiable (\emph{un}-identifiable), for example if calculating
a causal effect requires distributions of variables that we do not
observe, then we cannot use causal inference to calculate a causal
effect, even with an infinite sample of data.

% backdoor path}

A necessary condition for unidentifiability is the presence of an
unblocked backdoor path from cause to effect. Backdoor paths are any
paths going through parents of the cause to the effect. We can block
these paths by selectively observing variables such that no
information passes through them \citep{geiger-d-sep}. If we can
observe variables along the backdoor paths uch that they are blocked,
then we have satisfied the \emph{back-door criterion}
\citep{pearl2009} and we can calculate unbiased causal effects from
data.

\begin{figure}
  \noindent\includegraphics[]{./mutilated-cloud-aerosol.pdf}\\
  \caption{A mutilation of Figure \ref{fig:toy}, where we have removed
    directed causal paths from the cause (cloud) to the effect (surface
    solar radiation). We can see that there is covariability between
    cloud and surface solar radiation in the data, that is not
    explained by a causal connection between cloud and surface solar
    radiation, but instead is because of induced co-variability caused
    by aerosol.}
  \label{fig:mutilated-toy}
\end{figure}

% backdoor path with example}
Understanding backdoor paths and the backdoor criterion is helped by
example. Returning to our toy example (Figure \ref{fig:toy}), we will
attempt to calculate the causal effect of clouds on sunlight. In other
words, we want to isolate the variability of sunlight due to the
causal link from cloud to sunlight. However, aerosols both affect
cloud (edge from aerosol to cloud), and sunlight, so if we naively
calculate a causal effect we would get a biased estimate of the mean
effect of cloud on sunlight. To demonstrate, consider generated
data where the ``true'' SCM is:

\begin{equation}
  SCM =
  \begin{cases}
    f_{aerosol} &: U_{aerosol} \sim \text{uniform (0, 1]}\\
    f_{cloud} &: Cloudy | Clear \sim \text{bernoulli(aerosol)}     \\
    f_{sunlight} &: \begin{cases}
      Cloudy &: 0.6 \cdot \text{clear sky radiation}  \\
      Clear &: \text{clear sky radiation}
    \end{cases}
  \end{cases}
\end{equation}

where:

\begin{equation*}
  \text{clear sky radiation} = U_{sunlight} \cdot (1 - aerosol); \;
  U_{sunlight} \sim \text{Normal(340, 30)}
\end{equation*}

Now, consider not knowing the underlying SCM, but just receiving
passive data of cloud and sunlight data. If one were interested in
calculating the effect of cloud on sunlight, and aerosol data were not
available or one where very naive, one approach would be to bin the
data by cloudy and clear conditions and comapre the amount of sunlight
between cloudy and clear observations (Figure
\ref{fig:naive-cloud-sunlight}). This approach suggests that clouds
reduce sunlight by, on average, 160 W m$^{-2}$, which is a massive
overestimation of the true average effect of clouds derived from the
SCM of -68 W m$^{-2}$. Aerosol induces co-variability between cloud
and aerosol that unrelated to the causal link from cloud to
aerosol. Graphically, this is clarified by removing all edges from our
cause (cloud) to children of our cause (in this case sunlight) (Figure
\ref{fig:mutilated-toy}). We see that cloud is not independent of
surface solar radiation in the mutilated graph.  However if aerosol
were fixed (e.g. observed or not varying), cloud and sunlgiht would be
independent of each other in Figure \ref{fig:mutilated-toy}, and
in(Figure \ref{fig:toy}) all co-variability between cloud and sunlight
would be only due to the causal edge between cloud and sunlight.
Mathematically incorporating this requirement that we must condiiton
on aerosol to isolate the causal effect of cloud on radiation gives
the identification of the causal effect of cloud and aerosol according
to the backdoor criterion:

\begin{figure}
  \includegraphics[]{naiveCloudSunlight.pdf}
  \caption{A naive approach to estimating the ``effect'' of clouds on
    sunlight: bin observations by cloudy and clear day, and compare
    the values of sunlight. This approach yields a massive
    overestimation of the true causal effect of clouds on sunlight,
    which is -68.0 W m$^{-2}$}
  \label{fig:naive-cloud-sunlight}
\end{figure}

\begin{equation}
  P(S | do(C = c)) = \int_{a} P(S \, | \, C = c,
  A=a) \, P(A=a) \; da,
  \label{eq:3}
\end{equation}

where the \textit{do}-calculus \citep{pearl2009} term
($P(S \, | \, do(C\, = \,c))$) represents the probability of sunlight
if we did an experiment where we intervened and set cloud to a value
of our choosing (in this case $c$). In the case that observations of
aerosols are not available, our causal effect is not identifiable and
we cannot use causal inference no matter how large the sample sizes of
clouds and aerosols. This is a powerful time saving research tool:
without having to manipulate data or estimate marginal or conditional
distributions, we can determine whether it is possible to calculate a
causal effect of interest, given the available observations and our
assumptions about the causal dependencies in the system.  We later use
this theory to theoretically assess which general problems are
tractable in earth science using causal inference (Section
\ref{sec:necess-cond-caus}).

Once we have established that a causal effect is identifiable from
data, we must use estimate the required observational distributions
(Equation (\ref{eq:3}) from data. Often it may be more computationally
tractable to calculate an average causal effect, rather than the full
causal distribution $P(S | do(C=c))$. Returning to our toy example
(Figure \ref{fig:toy}), the average effect is defined as:

\begin{equation}
  \mathbb{E}(S | do(C = c)) = \int_{s} s \, P(S = s
  | do(C=c)) \, ds,
  \label{eq:4}
\end{equation}

Substituting Equation (\ref{eq:3}) into Equation (\ref{eq:4})
rearranging gives:

\begin{equation}
  \mathbb{E}(S | do(C = c))  = \int_{a} P(A=a) \; \mathbb{E}(S=s |
  C=c, A=a) \, d a,
  \label{eq:5}
\end{equation}

Where $\mathbb{E}(S=s \, | \, C=c, A=a)$ is just a regression of sunlight on
cloud and aerosol. Estimating the marginal $P(A)$ is difficult, but
if we assume that our observations are independent and identically
distributed (IID) and we have a large enough sample, we can use the
law of large numbers to approximate Equation (\ref{eq:5}) with:

\begin{equation}
  \mathbb{E}(S | do(C = c))  \approx \frac{1}{n} \sum_{i=1}^n \mathbb{E}(S=s_i |
  C=c, A=a_i).
  \label{eq:6}
\end{equation}

Data or prior knowledge can inform the regression function for
$\mathbb{E}(S=s_i | C=c, A=a_i)$. In the toy example, a linear model
conditional on cloud appears to be a good choice of regression
function (Figure \ref{fig:linear}). The causal effect of clouds on
sunlight as calculated using Equation (\ref{eq:6})) (e.g.
$\mathbb{E}(S | do(C = \text{cloudy})) - \mathbb{E}(S | do(C =
\text{clear}))$) is -68.52 W m$^{-2}$, which closely matches the true
causal effect from the SCM of -68 W m$^{-2}$. This example
demonstrates how causal inference and theory can be used to calulate
unibased average effects using regression. Further, causal inference
can be used to justify and communicate assumptions in any
observational analyses employing regression. In the best case, the
causal effect is indentifiable from the availalbe observations, and
the regression analysis can be framed as an average causal effect. In
the worst case that identification is not possible from the available
observations, one may present the regression as observed associations
between variables. However, presentation of a causal graph still aids
the reader: the reader can see from the causal graph what the
confounders and unobserved sources of covariability are between the
predictors and the output. In all cases, the presentation of a causal
graph makes explicit the assumptions about the causal dependencies of
the system. Wherever possible, we commend including causal graphs with
any observation-based analyses.

In summary of the main points of this introduction to causal graphical
models, structural causal models, and \textit{do-}calculus:

\begin{itemize}
\item Graphical causal models encode our assumptions about causal
  dependencies in a system (edges are drawn \emph{from} causes
  \emph{to} effects).
\item Each graphical causal model corresponds to a structural causal
  model, which are a set of functions that map causes and random
  variations to effects.
\item In order to calculate an unbiased causal effect from data, we
  must remove all covariability between our cause and effect that is
  not due to the directed causal path from cause to effect. The
  presence of non-causal covariation between the cause and effect can
  be deduced from the causal graph: the presence of an unblocked
  backdoor path from the cause to the effect leads to non-causal
  covariation.
\item The backdoor criterion identifies the distributions we must
  calculate from data in order to block a backdoor path, remove
  non-causal covariability between the cause and effect, and
  calculate an unbiased causal effect from data.
\item The \emph{average} causal effect can be reliably approximated
  with regression (Equation (\ref{eq:6})) derived from the backdoor
  criterion. In this scenario, causal theory and graphs identify the
  variables that should (and should not be) included in the
  regression to calculate an unbiased causal effect.
\item Causal identification is a flexible tool that provides the
  distributions that must be estimated by data, while making no
  assumptions about the forms of those distribution. However,
  parametric assumptions can be applied to make the calculation of
  those distributions from data more computationally tractable.
\end{itemize}

\begin{figure}
  \includegraphics[]{aerosolSunlight.pdf}
  \caption{A linear relationship between aerosol and sunlight,
    conditional on cloud. If we use linear regression to calculate the
    average causal effect of cloud on sunlight, as in Equation
    (\ref{eq:6}), our result is very close to the true causal effect
    of -68.0 W m$^{-2}$.}
  \label{fig:linear}
\end{figure}

Here we focused on the backdoor criterion to block backdoor paths. An
un-blockable backdoor path from the cause to the effect is a necessary
condition for un-unidentifiability. However, it is not sufficient
(e.g. there are other identification strategies like the front door
criterion and instrumental variables that do not rely on observing
variabiles along the backdoor path). For a complete discussion of
sufficient conditions for un-unidentifiability we refer you to
\citet{shpitser2006}. For the purpose of identifying generally
tractable causal inference approaches in earth science (Section
\ref{sec:necess-cond-caus}), we focus the backdoor criterion. Given
that the earth system evolves continuously according to an underlying
dynamical system, and that we only partially observe the state space
(Section \ref{state-space}), these other identification strategies are
not applicable, and an unblockable backdoor path is a sufficient
condition for un-unidentifiability in the types of graphs that are
representative of earth science systems (Figure \ref{fig:generic}).

\textit{/TODO: need to formally prove this or adjust language/}

\section{Necessary conditions for causal identification in the earth
  systems}
\label{sec:necess-cond-caus}

Earth science systems are dynamical system evolving through time
according to an underlying system state. This offers both advantages
and challenges for causal inference. Challenges involve our partial
observation of the system's state space, while advantages include the
temporal ordering of events; we know that future events can have no
causal effect on the past.

Causal identification and tractable causal inference to earth science
requires assumptions about the unobserved portion of the state
space. For example, without assumptions we know that the unobserved
portion of the state space will introduce confounding for any causal
affect of interest (Figure \ref{fig:generic}). We do not observe the
state space at every time (e.g. $S(t-1/2)$ in Figure
\ref{fig:generic}), and at any given time, we do not observe the state
space at all locations and for all state variables (e.g. $S(t)$ and
$S(t-1)$ in Figure \ref{fig:generic}). So, if we are interested in the
causal effect of any state variable on some a variable ($E$) at time
$t+1$, then the causal effect will be confounded by the unobserved
portions of the state space, and calculating a causal effect is
impossible (un-identifable) without additional assumptions. We will
apply causal graph theory to identify such additional assumptions that
would make the caulcations of causal effects tractable, and may be
reasonable in many earth science scenarios.

\begin{figure}
  \includegraphics[]{./generic-graph.pdf}
  \caption{A generic graph of the earth system state sequence, limited
    to a 3 time sequence subset of the infinite sequence. Unobserved
    nodes are outlined by dashed lines.  In the scenario that we are
    interested in calculating the causal effect of any portion of the
    state space at time $t$ on some effect ($E$) at time $t+1$, the
    causal effect will be confounded by the unobserved portions of the
    state space, and calculating the causal effect is impossible
    (un-identifiable) without additional assumptions.}
  \label{fig:generic}
\end{figure}

\subsection{Statistical reconstruction of the state space with time
  lagged observations}
\label{sec:stat-reconstr-state}

Takens's theorem implies that we can reconstruct the unobserved
portions of the state space using lagged observations back in time. In
this case, the causal graph is greatly simplified (Figure
\ref{fig:reconstructed}). If we assume that the state is
reconstructable, then we can calulcate the causal effect of any
reconstructed state on any future variable or process. The primary
advantages of this approach is that it is the minimum assumption
required to caclulate a causal effect in the generic earth system
proposed in Figure \ref{fig:generic}. However there is no free lunch:
the significant disadcantage of this approach is that we cannot
examine the causal effect of a specific variable (e.g. soil moisture
at time $t$) on the efffect; we limit ourselves to only examining the
causal effect of changes in the entire state
holistically. Additionally, this approach requires a shift in
perspective from the relatively straightforward interpretation of
interventions on observable variables considered thus far. For
example, say that we reconstructed our state space from available
observations, and this reconstruction reduced to a discrete state
space of 10 states, with associated patterns in the observations at
time $\leq t$. If we are interested in the effect of a change in state on
a process at time $t+1$, for example a change from state $1$ to state
$2$, this correspondes to an intervention on the \textit{entire state
  space at time $t$}. This includes the unobserved portions of the
state space. So, our intervention no longer consists of just
intervenning on our observed variables to change them from state $1$
to state $2$, but instead intervenning on our observed variables
\emph{and all unobserved variables consistent with the system and
  changes in the observed variables from state $1$ to $2$.}
Conceptualizing unknowable changes in unobserved variables represents
a singificant barrier to interpretation, and motivates the exploration
of stronger assumptions that result in a clearer interetation (at the
cost of an increased likelihood that the assumptions are voilated).

In practice we also do not know the number of lagged observations of
the state space that are required for reconstruction at time
$t-1$. One approach is to use the data for guidance. This involves
predicting the the observations at time $t$ with iteratively
increasing numbers of lags. When the addition of more time lags does
not improve the prediction of variables at time $t$, we have
confidence that the relevant portions of state space at time $t=t-1$
has been reconstructed by the time series. However even this approach
can be problematic; while predictive power may plateau as we add more
temporal observations back in time, we could still gain causally
relevant knowledge going back further in time. For example, a
significant drought last year may impact ecosystem state in the
current year, but predicitive power may locally plateau going back in
time 2 months. This example highlights the challenges associated with
any method relying on state space reconstruction with lagged
observations: to be sure we reconstruct the state space we must
include observations from a potentially infinite temporal extent.

\begin{figure}
  \includegraphics[]{reconstruction.pdf}
  \caption{The causal graph under an assumption that we can
    reconstrcut the state space, for a two time step sequence. Each
    node S(t) is reconstructed from the observable potions of the
    state space at lagged times $< t$. We can examine the causal
    effect of the reconstructed state on any future observed process,
    represented by $E(t+1)$.}
  \label{fig:reconstructed}
\end{figure}

\subsection{Missing temporal observations induce independent random
  variations in cause and effect}
\label{sec:miss-temp-observ}

If we can assume that the missing temporal observations
(e.g. $S(t-1/2)$ in Figure \ref{fig:generic}) induce independent
random variations in the cause and effect, conditional on $S(t-1)$,
then the causal graph reduces to Figure \ref{fig:no-temporal}. In this
case, if we can reconstruct the state space at time $t-1$ using lagged
observations at time t $\leq t-1$, then we can block all backdoor
paths between our cause $C(t)$ and any future effect $E(t+1)$. In this
case, the causal effect would be calculated as:

\begin{equation}
  P(E(t+1)| do(C(t)=c)) = \int_{S(t-1)} P(E(t+1) \, | \, C(t)=c,
  S(t-1) = s
  )\; P(S(t-1)=s) \, d s,
\end{equation}

where $S(t-1)$ is reconstructed from time lags of observations at
times $\leq t-1$. From the graph it may appear that we could also
block backdoor paths by conditioning on $S'(t)$. However, under these
assumptions $S'(t)$ is incalculable because if we attempt to
reconstruct $S'(t)$ using lagged observations of the state space, the
reconstruction will estimate $S$ rather than $S'$, which includes
$C(t)$ because $C$ is a part of the state space. So, our cause will be
shadowed by our reconstruction of the state space and our causal
effect will be biased. We further must assume that the missing
temporal observation induce independent random variations in the cause
and effect; otherwise there would be an open backdoor path between the
cause and effect through the unobserved time slice. Relative to
Section \ref{sec:stat-reconstr-state}, the advantage of this approach
to causal inference is that we can calculate the causal effects of
individual observations (e.g. soil moisture), rather than needing to
interpret the causal impact of the entire state holistically. However,
this approach requires an additional assumption that the missing
temporal observations do not induce dependencies between the cause and
effect of interest. As with Section \ref{sec:stat-reconstr-state}, we
must be able to reconstruct the state space, which may not always be
possible, and may also muddy interpretation.

\begin{figure}
  \includegraphics[]{no-temporal.pdf}
  \caption{The causal graph under an assumption that missing temporal
    observations (e.g. $S(t-1/2)$ in  Figure \ref{fig:generic}) induce
    indepednent random variations in the cause ($C(t)$) and effect
    ($E(t+1)$). $S'(t)$ denotes the state space, not including the
    cause $C(t)$.}
  \label{fig:no-temporal}
\end{figure}

\subsection{The unobserved portion of the state space does not affect
  the cause}
\label{sec:observ-port-state}

If we assume that we observe all portions of the state space that
affect the effect, then we can calculate the effect of any
state variable on future processes. In this case, the graph is as in
Figure \ref{fig:observed}. If we can assume that there are no
interactions between observations at time $t$ then we can calulcate
the causal effect as:

\begin{equation}
  P(E(t+1)| do(C(t)=c)) = \int_{Obs'(t)} P(E(t+1) \, | \, C(t)=c,
  Obs'(t) = o
  )\; P(Obs'(t-1)=o) \, d \, o,
\end{equation}

where $Obs'(t)$ are all observations at time $t$, not including the
cause $C(t)$. Whether or not there are interactions between
observations at time $t$ is a strong function of the temporal and
spatial extent of the observations. If observations are instanteneous
point observatins precisiely indexed in time, then interactions
between them can likely be ignored. However, often the spatial and/or
temporal extents of observations overlap, and in this case an
assumption of zero interactions between observations is not
justified. In this case, we can still block backdoor paths by
conditioning on $Obs(t-1)$:

\begin{equation}
  P(E(t+1)| do(C(t)=c)) = \int_{Obs(t-1)} P(E(t+1) \, | \, C(t)=c,
  Obs(t-1) = o
  )\; P(Obs(t-1)=o) \, d \, o.
\end{equation}

The assumption that we observe all variables relevant to an effect is
a strong assumption. However, it provides significant benefits both in
terms of the statistical complexity of estimating the causal effect,
as well as in intepretation of the causal effect. We do not need to
reconstruct the state space  using lagged observations, which can be a
statistically and computationally challenging problem. Additionally,
the assumption that we observe everything relevant to the effect is
much easier to interpret relative to an assumption about the degree to
which we can (or cannot) reconstruct the relevant state space with
lagged observations.

\begin{figure}
  \includegraphics[]{observe-everything.pdf}
  \caption{The causal graph under an assumption that we observe all
    portions of the state causally relevant to the effect. $Obs'$ are
    all of the observations excluding $C$.}
  \label{fig:observed}
\end{figure}

\subsection{Applications at the human-earth system interface: when the
  cause is approximately independent of the system}
\label{human}

Causal inference becomes substantially more tractable under an
assumption that our causes of interest are independent from the
evolution of the state space (Figure \ref{fig:forcing}). While the
earth system certainly affects all processes on earth, in some cases
this may still be a reasonable assumption. For example, recent human
history demonstrates that global human actions are relatively
independent of the climate state \citep{arto2014drivers}. That is, we
have failed to reduce green house gas emissions even as global
temperature increased. Instances of reduced rises in global green
house gas emissions are usually due to global economic recession
(e.g. the 2008 financial crisis) rather than factors directly tied to
the climate state. In this example many global social, political and
economic factors are the primary causes of global green house gas
emission, and while the climate system may effect these factors, the
historical evidence suggests that the climate system exerts a
relatively small impact (with tragic consequences)
\citep{arto2014drivers}.

\emph{/discuss data problem? - we know
  don't observe the future of green house gas emissions/}.

\begin{figure}
  \includegraphics[]{./forcing-graph.pdf}
  \caption{A generic graph asserting an assumption that there are
    forcing external to the evolution of the state-space}
  \label{fig:forcing}
\end{figure}

Generally this logic applies to many scenarios on the "human-climate"
interface, such as land-use land-cover change in urban centers where
urban planning is relatively independent of recent climate
history. The general graph in Figure \ref{fig:forcing} is
representative of many such scenarios at the human-climate interface,
and because the state space does not affect the cause, there are no
unblocked backdoor paths through the unobserved portions of the state
space. Causal inference is particularly tractable for this class of
problems.

\section{Relation to previous work and ``causal discovery''}
\label{sec:discovery}
The term ``causal inference'' has been used to describe two
techniques:

\begin{enumerate}
\item \textbf{Causal effect inference}: Calculating the causal effect
  of some processes on other, given data and assumptions about the
  causal structure of the system.
\item \textbf{Causal structure discovery}: Inferring the causal
  structure (e.g. the causal graph) of the system using data.
\end{enumerate}

This paper focuses on causal inference as (1): calculating causal
effects from data. Inferring the causal structure of the system (2),
is generally much more difficult and requires more
assumptions. However, there are other reasons to focus on (1) rather
than (2) in earth science: often in earth science we know or have a
strong a priori belief about the causal graph of our system. For
example, in the climate system we can identify the state variables and
the state at time \(t\) determines the state at time \(t+1\), even if
the exact functional form of the dependency through time is not
known. Therefore, we can write down a causal graph and do not need to
infer graph structure from data (Section \ref{sec:necess-cond-caus}).

\textit{/ TODO: tactfully critique causal discovery. a messy start is
  here:}

\textit{
  However, there has been considerable work and effort in applying
  causal structure discovery (2) in earth science
  \citep[e.g.,][]{ebert-uphoff2012,
    samarasinghe-casuality,runge-causal-timeseries,runge2019inferring}. A
  common application and motivation of these efforts is to filter and
  ignore causal links in the system through structure discovery. The
  ignored links are a function of the significance parameters used in
  the causal discovery algorithms. Causal inference of effects, as we
  explore here, represents a different approach where we do not rely on
  assumptions about the significance of effects, but instead make
  explicit assumptions about the causal structure of the system. To
  reiterate, given our domain knowledge of the earth science system we
  generally have high confidence about the causal structure of the
  system, and as we will see (Section \ref{sec:necess-cond-caus}) we can
  construct quite general graphs that are faithful to our knowledge
  about how dynamical systems evolve. It is possible some links in these
  graphs correspond to small effects, and these links would be removed
  through causal structure discovery. However, directly interpreting
  what constitutes a ``negligibly small'' effect calculation presented in
  physical units and probabilities, as calculated with causal effect
  inference, may be more transparent than interpreting missing links
  derived from causal structure discovery significance parameters. In
  other words, direct calculation of effects may be more transparent and
  interpretable for many readers, relative to a causal graph derived
  from significance parameters with more abstract meaning. We hope to
  motivate further research effort in causal effect inference to match
  recent efforts in causal structure discovery. Ideally causal effect
  inference and causal structure discovery will co-evolve as
  complementary abstractions for causal interpretation; researchers and
  readers can choose the method that suits their assumptions and
  needs./}

\textit{
  TODO: talk to Elias about ``no mixing'' theorem; from what he
  described it seems like it leads to the invalidation of ``causal
  discovery''
}

\section{Discussion}

\emph{TODO: expand these bullets into a full section}.

\emph{/and add discussion of the role of randomness vs predictive
  power, causes vs effects in an evolving state space system?/}

\begin{itemize}
\item Causal inference from data is a new tool with the potential to
  complement traditional scientific methods, including numerical and
  real world experimentation. In earth science, numerical models rely
  on approximations that deviate their behavior from reality, and real
  world experimentation may be intractable or unethical. Causal
  inference is a third tool to calculate the effects of physical
  processes that has a different set of advantages and
  disadvantages, and is a powerful complement to numerical and real
  world experimentation.
\item The characteristics of earth science systems offer advantages
  for causal inference: the temporal ordering of events and bounds on
  the propagation speed of information in the system allows us to
  filter which events that can affect a process, and identify the
  necessary portions of the state space we must condition on to block
  backdoor paths.
\item Applying causal inference to earth science systems also presents
  challenges: we only ever partially observe the state space of the
  system, so we must reconstruct the full state space using time
  lagged observations. This increases the computational and
  statistical complexity of the problem, and can make some application
  intractable.
\item Given these generic characteristics of earth science systems, we
  have identified two scenarios that meet the necessary conditions for
  applying causal inference:
  \begin{enumerate}
  \item The state space of the system is reconstructable
    from lagged observations of the system, as allowed by
    Takens's theorem, or
  \item The cause of interest can be assumed to be
    independent of the evolution of the system's state
    (e.g. forcing).
  \end{enumerate}
\end{itemize}

% /discuss transportability?/

% /discuss relationship of these ideas to causal discovery (critique
% causal discovery)?/

% /discuss issues with data available and observed range
% (e.g. generalization for cliamte research)?/

\bibliography{references.bib}

\end{document}